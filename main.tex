%%%%%%%%%%%%%%%%%%%%%%%%%%%%%%%%%%%%%%%%%
% The Legrand Orange Book
% LaTeX Template
% Version 2.2 (30/3/17)
%
% This template has been downloaded from:
% http://www.LaTeXTemplates.com
%
% Original author:
% Mathias Legrand (legrand.mathias@gmail.com) with modifications by:
% Vel (vel@latextemplates.com)
%
% License:
% CC BY-NC-SA 3.0 (http://creativecommons.org/licenses/by-nc-sa/3.0/)
%
% Compiling this template:
% This template uses biber for its bibliography and makeindex for its index.
% When you first open the template, compile it from the command line with the 
% commands below to make sure your LaTeX distribution is configured correctly:
%
% 1) pdflatex main
% 2) makeindex main.idx -s StyleInd.ist
% 3) biber main
% 4) pdflatex main x 2
%
% After this, when you wish to update the bibliography/index use the appropriate
% command above and make sure to compile with pdflatex several times 
% afterwards to propagate your changes to the document.
%
% This template also uses a number of packages which may need to be
% updated to the newest versions for the template to compile. It is strongly
% recommended you update your LaTeX distribution if you have any
% compilation errors.
%
% Important note:
% Chapter heading images should have a 2:1 width:height ratio,
% e.g. 920px width and 460px height.
%
%%%%%%%%%%%%%%%%%%%%%%%%%%%%%%%%%%%%%%%%%

%----------------------------------------------------------------------------------------
%	PACKAGES AND OTHER DOCUMENT CONFIGURATIONS
%----------------------------------------------------------------------------------------

\documentclass[11pt,fleqn]{book} % Default font size and left-justified equations

%----------------------------------------------------------------------------------------

\input{structure} % Insert the commands.tex file which contains the majority of the structure behind the template

\begin{document}

%----------------------------------------------------------------------------------------
%	TITLE PAGE
%----------------------------------------------------------------------------------------

\begingroup
\thispagestyle{empty}
\begin{tikzpicture}[remember picture,overlay]
\node[inner sep=0pt] (background) at (current page.center) {\includegraphics[width=\paperwidth]{background}};
\draw (current page.center) node [fill=ocre!30!white,fill opacity=0.6,text opacity=1,inner sep=1cm]{\Huge\centering\bfseries\sffamily\parbox[c][][t]{\paperwidth}{\centering The Search for a Title\\[15pt] % Book title
{\Large A Profound Subtitle}\\[20pt] % Subtitle
{\huge Dr. John Smith}}}; % Author name
\end{tikzpicture}
\vfill
\endgroup

%----------------------------------------------------------------------------------------
%	COPYRIGHT PAGE
%----------------------------------------------------------------------------------------

\newpage
~\vfill
\thispagestyle{empty}

\noindent Copyright \copyright\ 2013 John Smith\\ % Copyright notice

\noindent \textsc{Published by Publisher}\\ % Publisher

\noindent \textsc{book-website.com}\\ % URL

\noindent Licensed under the Creative Commons Attribution-NonCommercial 3.0 Unported License (the ``License''). You may not use this file except in compliance with the License. You may obtain a copy of the License at \url{http://creativecommons.org/licenses/by-nc/3.0}. Unless required by applicable law or agreed to in writing, software distributed under the License is distributed on an \textsc{``as is'' basis, without warranties or conditions of any kind}, either express or implied. See the License for the specific language governing permissions and limitations under the License.\\ % License information

\noindent \textit{First printing, March 2013} % Printing/edition date

%----------------------------------------------------------------------------------------
%	TABLE OF CONTENTS
%----------------------------------------------------------------------------------------

%\usechapterimagefalse % If you don't want to include a chapter image, use this to toggle images off - it can be enabled later with \usechapterimagetrue

\chapterimage{chapter_head_1.pdf} % Table of contents heading image

\pagestyle{empty} % No headers

\tableofcontents % Print the table of contents itself

\cleardoublepage % Forces the first chapter to start on an odd page so it's on the right

\pagestyle{fancy} % Print headers again

%----------------------------------------------------------------------------------------
%	PART
%----------------------------------------------------------------------------------------

\part{Part One}

%----------------------------------------------------------------------------------------
%	CHAPTER 1
%----------------------------------------------------------------------------------------

\chapterimage{chapter_head_2.pdf} % Chapter heading image

\chapter{Logika Matematika}

\section{Pernyataan Berkuantor}\index{Pernyataan Berkuantor}

Kuantor dari suatu pernyataan adalah istilah yang digunakan untuk menyatakan “berapa banyak” objek di dalam suatu kalimat atau pembicaraan. Selain untuk menyatakan kuantifikasi, kuantor juga biasa digunakan untuk mengubah kalimat terbuka menjadi suatu kalimat deklaratif.

Definisi : Suatu fungsi pernyataan adalah suatu kalimat terbuka di dalam semesta pembicaraan (semesta pembicaraan diberikan secara eksplisit atau implisit).
Perhatikan dua pernyataan berikut:
1.    Semua planet dalam sistem tata surya mengelilingi matahari.
2.    Ada ikan di laut yang menyusui.
Pernyataan yang mengandung kata semua atau setiap seperti pada pernyataan (1) disebut pernyataan berkuantor universal (kuantor umum). Ungkapan untuk semua atau untuk setiap, disebut kuantor universal atau kuantor umum. Sedangkan pernyataan yang mengandung kata ada atau beberapa seperti pada pernyataan (2) disebut pernyataan berkuantor eksistensial (kuantor khusus). Ungkapan beberapa atau ada disebut kuantor eksistensial atau kuantor khusus.

%------------------------------------------------

\section{Pernyataan Penyangkal (Lingkaran)}\index{Pernyataan Penyangkal  (Lingkaran)}

Dari sebuah pernyataan tunggal (atau majemuk), kita bisa membuat sebuah pernyataan baru berupa “ingkaran” dari pernyataan itu. “ingkaran” disebut juga “negasi” atau “penyangkalan”. Ingkaran menggunakan operasi uner (monar) “” atau “”.

Jika suatu pernyataan p benar, maka negasinya p salah, dan jika sebaliknya pernyataan p salah, maka negasinya p benar.

Perhatikan cara membuat ingkaran dari sebuah pernyataan serta menentukan nilai kebenarannya!

1. p         : kayu memuai bila dipanaskan (S)

~ p      : kayu tidak memuai bila dipanaskan (B)

2.  r          : 3 bilangan positif (B)

~ r        : (cara mengingkar seperti ini salah)

3 bilangan negative

(Seharusnya) 3 bukan bilangan positif  (S)

Nilai kebenaran

Jika p suatu pernyataan benilai benar, maka  ~p bernilai salah dan sebaliknya jika p bernilai salah maka ~p bernilai benar

Konjungsi 

Gabungan  dua  pernyataan  tunggal  yang  menggunakan  kata penghubung  “dan”  sehingga  terbentuk  pernyataan majemuk  disebut konjungsi. Konjungsi mempunyai kemiripan dengan operasi irisan () pada  himpunan.  Sehingga  sifat-sifat  irisan  dapat  digunakan  untuk mempelajari  bagian  ini.



%------------------------------------------------

\section{Penarikan Kesimpulan}\index{Penarikan Kesimpulan}

Lists are useful to present information in a concise and/or ordered way\footnote{Footnote example...}.

\subsection{Numbered List}\index{Lists!Numbered List}

\begin{enumerate}
\item The first item
\item The second item
\item The third item
\end{enumerate}

\subsection{Bullet Points}\index{Lists!Bullet Points}

\begin{itemize}
\item The first item
\item The second item
\item The third item
\end{itemize}

\subsection{Descriptions and Definitions}\index{Lists!Descriptions and Definitions}

\begin{description}
\item[Name] Description
\item[Word] Definition
\item[Comment] Elaboration
\end{description}

%----------------------------------------------------------------------------------------
%	CHAPTER 2
%----------------------------------------------------------------------------------------

\chapter{Induksi Matematika}

\section{Metode Pembuktian Langsung dan Tidak Langsung}\index{Metode Pembuktian Langsung dan Tidak Langsung}

This is an example of theorems.

\subsection{Several equations}\index{Theorems!Several Equations}
This is a theorem consisting of several equations.

\begin{theorem}[Name of the theorem]
In $E=\mathbb{R}^n$ all norms are equivalent. It has the properties:
\begin{align}
& \big| ||\mathbf{x}|| - ||\mathbf{y}|| \big|\leq || \mathbf{x}- \mathbf{y}||\\
&  ||\sum_{i=1}^n\mathbf{x}_i||\leq \sum_{i=1}^n||\mathbf{x}_i||\quad\text{where $n$ is a finite integer}
\end{align}
\end{theorem}

\subsection{Single Line}\index{Theorems!Single Line}
This is a theorem consisting of just one line.

\begin{theorem}
A set $\mathcal{D}(G)$ in dense in $L^2(G)$, $|\cdot|_0$. 
\end{theorem}

%------------------------------------------------

\section{Kontradiksi}\index{Kontradiksi}

This is an example of a definition. A definition could be mathematical or it could define a concept.

\begin{definition}[Definition name]
Given a vector space $E$, a norm on $E$ is an application, denoted $||\cdot||$, $E$ in $\mathbb{R}^+=[0,+\infty[$ such that:
\begin{align}
& ||\mathbf{x}||=0\ \Rightarrow\ \mathbf{x}=\mathbf{0}\\
& ||\lambda \mathbf{x}||=|\lambda|\cdot ||\mathbf{x}||\\
& ||\mathbf{x}+\mathbf{y}||\leq ||\mathbf{x}||+||\mathbf{y}||
\end{align}
\end{definition}

%------------------------------------------------

\section{Induksi Matematis}\index{Induksi Matematis}

\begin{notation}
Given an open subset $G$ of $\mathbb{R}^n$, the set of functions $\varphi$ are:
\begin{enumerate}
\item Bounded support $G$;
\item Infinitely differentiable;
\end{enumerate}
a vector space is denoted by $\mathcal{D}(G)$. 
\end{notation}

%------------------------------------------------

\section{Remarks}\index{Remarks}

This is an example of a remark.

\begin{remark}
The concepts presented here are now in conventional employment in mathematics. Vector spaces are taken over the field $\mathbb{K}=\mathbb{R}$, however, established properties are easily extended to $\mathbb{K}=\mathbb{C}$.
\end{remark}

%------------------------------------------------

\section{Corollaries}\index{Corollaries}

This is an example of a corollary.

\begin{corollary}[Corollary name]
The concepts presented here are now in conventional employment in mathematics. Vector spaces are taken over the field $\mathbb{K}=\mathbb{R}$, however, established properties are easily extended to $\mathbb{K}=\mathbb{C}$.
\end{corollary}

%------------------------------------------------

\section{Kontradiksi}\index{Kontradiksi}

This is an example of propositions.

\subsection{Several equations}\index{Propositions!Several Equations}

\begin{proposition}[Proposition name]
It has the properties:
\begin{align}
& \big| ||\mathbf{x}|| - ||\mathbf{y}|| \big|\leq || \mathbf{x}- \mathbf{y}||\\
&  ||\sum_{i=1}^n\mathbf{x}_i||\leq \sum_{i=1}^n||\mathbf{x}_i||\quad\text{where $n$ is a finite integer}
\end{align}
\end{proposition}

\subsection{Single Line}\index{Propositions!Single Line}

\begin{proposition} 
Let $f,g\in L^2(G)$; if $\forall \varphi\in\mathcal{D}(G)$, $(f,\varphi)_0=(g,\varphi)_0$ then $f = g$. 
\end{proposition}

%------------------------------------------------

\section{Examples}\index{Examples}

This is an example of examples.

\subsection{Equation and Text}\index{Examples!Equation and Text}

\begin{example}
Let $G=\{x\in\mathbb{R}^2:|x|<3\}$ and denoted by: $x^0=(1,1)$; consider the function:
\begin{equation}
f(x)=\left\{\begin{aligned} & \mathrm{e}^{|x|} & & \text{si $|x-x^0|\leq 1/2$}\\
& 0 & & \text{si $|x-x^0|> 1/2$}\end{aligned}\right.
\end{equation}
The function $f$ has bounded support, we can take $A=\{x\in\mathbb{R}^2:|x-x^0|\leq 1/2+\epsilon\}$ for all $\epsilon\in\intoo{0}{5/2-\sqrt{2}}$.
\end{example}

\subsection{Paragraph of Text}\index{Examples!Paragraph of Text}

\begin{example}[Example name]
\lipsum[2]
\end{example}

%------------------------------------------------

\section{Exercises}\index{Exercises}

This is an example of an exercise.

\begin{exercise}
This is a good place to ask a question to test learning progress or further cement ideas into students' minds.
\end{exercise}

%------------------------------------------------

\section{Problems}\index{Problems}

\begin{problem}
What is the average airspeed velocity of an unladen swallow?
\end{problem}

%------------------------------------------------

\section{Vocabulary}\index{Vocabulary}

Define a word to improve a students' vocabulary.

\begin{vocabulary}[Word]
Definition of word.
\end{vocabulary}
%----------------------------------------------------------------------------------------
%	CHAPTER 3
%----------------------------------------------------------------------------------------

\chapterimage{chapter_head_2.pdf} % Chapter heading image

\chapter{Pertidaksamaan Linear Dua Variabel}

\section{Pengertian Pertidaksamaan Linear Dua Variabel}\index{Pengertian Pertidaksamaan Linear Dua Variabel}

\section{Penerapan Pertidaksamaan Linear Dua Variabel}\index{Penerapan Pertidaksamaan Linear Dua Variabel}

%----------------------------------------------------------------------------------------
%	CHAPTER 4
%----------------------------------------------------------------------------------------

\chapterimage{chapter_head_2.pdf} % Chapter heading image

\chapter{Program Linear Dua Variabel}

\section{Pengertian Program Linear Dua Variabel}\index{Pengertian Program Linear Dua Variabel}

\section{Sistem Pertidaksamaan Linear Dua Variabel}\index{Sistem Pertidaksamaan Linear Dua Variabel}

\section{Nilai Optimum Fungsi Objektif}\index{Nilai Optimum Fungsi Objektif}
\subsection{Teorema}\index{Lists!Teorema}
Fungsi dalam pembuatan model matematika dinyatakan dalam bentuk z = ax + by. Bentuk inilah yang akan dioptimumkan baik itu minimum ataupun maksimum yang disebut sebagai fungsi objektif. Jadi, fungsi Objektif dari program linear adalah fungsi z = ax + by yang akan ditentukan nilai optimumnya.
Ada beberapa cara untuk mencari nilai optimum (maksimum atau minimum) fungsi objektif, di antaranya yaitu:
\begin{itemize}
\item Metode Uji Titik Pojok
\item Metode Garis Selidik
\end{itemize}
Dari kedua metode tersebut, metode yang paling mudah dan sering diterapkan adalah metode uji titik pojok.
Metode Uji Titik Pojok adalah suatu metode dengan mensubstitusikan titik-titik pojok pada suatu daerah himpunan penyelesaian (DHP) ke fungsi objektif. Nilai maksimum berarti nilai yang paling besar yang kita ambil, begitu juga sebaliknya untuk nilai minimum kita ambil paling kecil.


Untuk menentukan nilai optimum baik nilai minimum atau maksimum dengan metode uji titik pojok, kita bisa melakukan langkah-langkah di bawah ini:
\begin{enumerate}
\item Buatlah model matematikanya yang terdiri dari fungsi kendala dan fungsi tujuan.
\item Tentukan daerah himpunan penyelesaiiannya (DHP) dan titik pojoknya.
\item Substitusikan semua titik pojok ke fungsi tujuannya (fungsi objektifnya), dan tentukan yang diminta apakah nilai maksimum atau minimum.
\end{enumerate}
\begin{figure}[h]
\centering\includegraphics[scale=0.7]{dhp}
\caption{Daerah Hasil Penyelesaian (DHP)}
\end{figure}

Misalnya untuk gambar DHP di bawah ini kita bisa lihat bahwa titik pojok terdapat pada titik A, titik B, dan titik C.
\subsection{Contoh Soal}
\index{Examples!Contoh Soal}

\begin{example}
Tentukanlah nilai maksimum dan nilai minimum fungsi tujuan f(x,y) = 1500x + 1250y berdasarkan DHP berikut ini.
\begin{figure}[h]
\centering\includegraphics[scale=0.7]{contohsoaloptimum}
\end{figure}
\end{example}
Jawab:\\
Diketahui bahwa titik pojok yang terdapat pada gambar di atas adalah titik A,b,C dan O. Titik C belum mempunyai titik koordinat sehingga kita harus mencari terlebih dahulu dengan cara eliminasi kedua persamaan garis.\\
Menentukan titik C:
\begin{figure}[h]
\centering\includegraphics[scale=0.7]{eliminasi}
\end{figure}\\
Substitusi x=40 ke persamaan x + y = 60\\
x + y = 60\\
40 + y = 60\\
y = 20.\\
Sehingga titik C adalah C(40,20).\\
Substitusi semua titik pojok ke fungsi tujuan : f(x,y) = 1500x + 1250y.\\

\begin{tabular}{l l}
A(0,60) & f = 1500(0) + 1250(60) = 75000 \\
B(56,0) & f = 1500(56) + 1250(0) = 84000 \\
C(40,20) & f = 1500(40) + 1250(20) = 85000 \\
D(0,0) & f = 1500(0) + 1250(0) = 0 \\
\end{tabular}

Jadi fungsi f(x,y) = 1500x + 1250y di titik C(40,20) dengan nilai maksimumnya adalah f = 85000. Sedangkan untuk titik minimum f(x,y) = 1500x + 1250y di titik C(0,0) dengan nilai minimumnya adalah f = 0.

\subsection{Soal Latihan }\index{Lists!Soal Latihan}


\begin{exercise}

\begin{enumerate}
\item Tentukan nilai maksimum f(x,y) = 3x + 4y pada himpunan penyelesaian sistem pertidaksamaan berikut: $x + 2y \le 10, 4x + 3y \le 24, x \ge 0, y \ge 0.$
\item Tentukan nilai maksimum dan nilai minimum dari fungsi objektif z = 2x + 3y yang memenuhi $x + y \le 7, x \ge 0$, dan $y \ge 0, x,y \in R.$
\item Tentuan nilai maksimum Z = 2x + 5y dari daerah penyelesaian (daerah yang diarsir) pada gambar di bawah ini:\\
\includegraphics[scale=0.7]{soaloptimum}
\end{enumerate}
\end{exercise}

\section{Penerapan Program Linier Dua Variabel}\index{Penerapan Program Linier Dua Variabel}
Program linear banyak digunakan dalam kehidupan sehari-hari, misalnya dalam bidang ekonomi, perdagangan, dan pertanian.\\

Misalkan dalam bidang perdagangan atau pengusaha, para pedagang atau pengusaha tentu ingin memperoleh keuntungan maksimum. Sebelum melakukan transaksi ataupun pengambilan keputusan dalam usahanya, mereka pasti membuat perhitungan yang matang tentang langkah apa yang harus dilakukan. Oleh karena itu, diperlukan metode yang tepat dalam pengambilan keputusan pedagang atau pengusaha tersebut untuk memperoleh keuntungan maksimum dan meminimumkan kerugian yang mungkin terjadi.\\

Seperti yang telah disinggungkan pada section sebelumnya, ada beberapa metode untuk menyelesaikan masalah program linear dua variabel, di antaranya yang kita bahas adalah metode uji titik pojok. Nah untuk contoh soal yang diambil dari kehidupan sehari-hari sebagai berikut:\\
\\
\begin{example}
Ling ling membeli 240 ton beras untuk dijual lagi. Ia menyewa dua jenis truk untuk mengangkut beras tersebut. Truk jenis A memiliki kapasitas 6 ton dan truk jenis B memiliki kapasitas 4 ton. Sewa tiap truk jenis A adalah Rp 100.000,00 sekali jalan dan truk jenis B adalah Rp 50.000,00 sekali jalan. Maka Ling ling menyewa truk itu sekurang-kurangnya 48 buah. Berapa banyak jenis truk A dan B yang harus disewa agar biaya yang
dikeluarkan minimum?
\end{example}
Jawab:\\
Untuk menyelesaikan masalah ini kita akan menggunakan metode uji titik sudut yang telah kita pelajari pada section nilai optimum fungsi objektif.\\
Dari soal di atas dapat diperoleh bahwa:
\begin{figure}[h]
\centering\includegraphics[scale=0.7]{jawab1}
\end{figure}\\
Dan ketika kita ubah ke model matematika akan seperti berikut:\\
$x + y \ge 48,$\\
$6x + 4y \ge 240,$\\
$x \ge 0, y \ge 0,$ x, y anggota bilangan cacah\\
\\
Dengan fungsi objektifnya yaitu  f(x, y) = 100.000x + 50.000y.\\
\begin{figure}[h]
\centering\includegraphics[scale=0.6]{jawab2}
\end{figure}\\
Dari gambar di atas dapat kita ketahui bahwa titik  pojok dari daerah penyelesaian di atas adalah titik potong garis 6x + 4y = 240 dengan sumbu-y, titik potong garis x + y = 48 dengan sumbu-x, dan titik potong garis-garis x + y = 48 dan 6x + 4y = 240.\\
Titik potong garis 6x + 4y = 240 dengan sumbu-y adalah titik (0, 60). Titik potong garis x + y = 48 dengan sumbu-x adalah titik (48, 0). Sedangkan titik potong garis-garis x + y = 48 dan 6x + 4y = 240 dapat dicari dengan menggunakan cara eliminasi berikut ini.\\
\begin{figure}[h]
\centering\includegraphics[scale=0.7]{jawab3}
\end{figure}\\

Substitusi semua titik pojok ke fungsi tujuan : f(x,y) = 100000x + 50000y.\\

\begin{tabular}{l l}
A(0,60) & f = 1500(0) + 1250(60) = 3000000 \\
B(48,0) & f = 1500(48) + 1250(0) = 4800000 \\
C(24,24) & f = 1500(24) + 1250(24) = 3600000 \\
D(0,0) & f = 1500(0) + 1250(0) = 0 \\
\end{tabular}


Jadi Dari ketiga hasil tersebut, dapat diperoleh bahwa agar biaya yang dikeluarkan minimum, Ling ling harus menyewa 60 truk jenis B dan tidak menyewa truk jenis A.\\



%----------------------------------------------------------------------------------------
%	CHAPTER 5
%----------------------------------------------------------------------------------------

\chapterimage{chapter_head_2.pdf} % Chapter heading image

\chapter{Matriks}

\section{Pengertian Matriks}\index{Pengertian Matriks}

\section{Operasi Matriks}\index{Operasi Matriks}

\section{Determinan dan Invers Matriks Berorde 2x2 dan 3x3 }\index{Determinan dan Invers Matriks Berorde 2x2 dan 3x3 }

\section{Pemakaian Matriks Pada Pransformasi Geometri}\index{Pemakaian Matriks Pada Pransformasi Geometri}

%----------------------------------------------------------------------------------------
%	CHAPTER 6
%----------------------------------------------------------------------------------------

\chapterimage{chapter_head_2.pdf} % Chapter heading image

\chapter{Barisan dan Deret}

\section{Pola Bilangan}\index{Pola Bilangan}

\section{Barisan dan Deret Aritmatika}\index{Barisan dan Deret Aritmatika}
Pengertian Barisan Aritmatika
Sebelum memahami pengertian barisan aritmatika kita harus mengetahui terlebih dahulumengenai pengertian basiran bilangan. Barisan bilangan merupakan sebuah urutan dari bilangan yang dibentuk dengan berdasarkan kepada aturan-aturan tertentu. Edangkan barisan aritmetika dapat didefinisikan sebagai suatu barisan bilangan yang tiap-tiap pasangan suku yang berurutan mengandung nilai selisih yang sama persis, contohnya adalah barisan bilangan: 2, 4 , 6, 8, 10, 12, 14, …

Barisan bilangan tersebut dapat disebut sebagai barisana aritmatika karena masing-masing suku memiliki selisih yang sama yaitu 2. Nilai selisih yang muncul pada barisan aritmatika biasa dilambangkan dengan menggunakan huruf b. Setiap bilangan yang membentuk urutan suatu barisan aritmatika disebut dengan suku. Suku ke n dari sebuah barisan aritmatika dapat disimbolkan dengan lambang Un jadi untuk menuliskan suku ke 3 dari sebuah barisan kita dapat menulis U3. Namun, ada pengecualian khusus untuk suku pertama di dalam sebuah barisan bilangan, suku pertama disimbolkan dengan menggunakan huruf a.

Maka, secara umum suatu barian aritmatika memiliki bentuk :

U1,U2,U3,U4,U5,…Un-1
a, atb, a+2b, a+3b, a+4b,…a+(n-1)b

Cara Menentukan Rumus suku ke-n dari Sebuah Barisan
Pada barisan aritmatika, mencaru rumus suku ke-n menjadi lebih mudah karena memiliki nilai selisih yang sama, sehingga rumusnya adalah:

U2 = a + b
U3 = u2 + b = (a + b) + b = a + 2b
U4 = u3 + b = (a + 2b) + b = a + 3b
U5 = u4 + b = (a + 3b) + b = a + 4b
U6 = u5 + b = (a + 4b) + b = a + 5b
U7 = u6 + b = (a + 5b) + b = a + 6b
.
\section{Barisan dan Deret Geometri}\index{Barisan dan Deret Geometri}

%----------------------------------------------------------------------------------------
%	PART
%----------------------------------------------------------------------------------------

\part{Part Two}

%----------------------------------------------------------------------------------------
%	CHAPTER 7
%----------------------------------------------------------------------------------------

\chapterimage{chapter_head_1.pdf} % Chapter heading image

\chapter{Limit Fungsi Aljabar}

\section{Table}\index{Table}

\begin{table}[h]
\centering
\begin{tabular}{l l l}
\toprule
\textbf{Treatments} & \textbf{Response 1} & \textbf{Response 2}\\
\midrule
Treatment 1 & 0.0003262 & 0.562 \\
Treatment 2 & 0.0015681 & 0.910 \\
Treatment 3 & 0.0009271 & 0.296 \\
\bottomrule
\end{tabular}
\caption{Table caption}
\end{table}

%------------------------------------------------

\section{Figure}\index{Figure}

\begin{figure}[h]
\centering\includegraphics[scale=0.5]{placeholder}
\caption{Figure caption}
\end{figure}

%----------------------------------------------------------------------------------------
%	CHAPTER 8
%----------------------------------------------------------------------------------------

\chapterimage{chapter_head_2.pdf} % Chapter heading image

\chapter{Turunan Fungsi Aljabar}

\section{Pengertian Turunan}\index{Pengertian Turunan}

\section{Sifat-Sifat Turunan Fungsi Aljabar}\index{Sifat-Sifat Turunan Fungsi Aljabar}

\section{Penerapan Turunan Fungsi Aljabar}\index{Penerapan Turunan Fungsi Aljabar}

\section{Nilai-Nilai Stasioner}\index{Nilai-Nilai Stasioner}

\section{Aplikasi Turunan}\index{Aplikasi Turunan}
Konsep turunan adalah subjek yang banyak berperan dalam aplikasi matematika di kehidupan sehari-hari di berbagai bidang. Konsep turunan digunakan untuk
menentukan interval fungsi naik/turun, keoptimalan fungsi dan titik belok suatu kurva.
\subsection{Fungsi Naik dan Fungsi Turun}
Coba bayangkan ketika kamu pergi ke plaza atau mall, di sana kita temukan ekskalator atau lift. Gerakan lift dan ekskalator saat naik dapat diilustrasikan sebagai fungsi naik. Demikian juga gerakan lift dan ekskalator saat turun dapat diilustrasikan sebagai fungsi turun. Amatilah beberapa grafik fungsi naik dan turun di bawah ini dan coba tuliskan cirri-ciri fungsi naik dan fungsi turun sebagai ide untuk mendefinisikan fungsi naik dan turun.

Beberapa grafik fungsi turun dari kiri ke kanan

\includegraphics[width=6cm,height=3cm]{naikturun1.png}

Beberapa grafik fungsi naik dari kiri ke kanan

\includegraphics[width=6cm,height=3cm]{naikturun2.png}

Dari beberapa contoh grafik fungsi naik dan turun di atas,mari kita definisikan fungsi naik dan turun sebagai berikut.

\includegraphics[width=10cm,height=3cm]{naikturun3.png}

\includegraphics[width=10cm,height=7cm]{naikturun4.png}

\subsection{Aplikasi Turunan dalam Permasalahan Fungsi Naik dan Fungsi Turun}

Mari kita bahas aplikasi turunan dalam permasalahan fungsi naik dan fungsi turun dengan memperhatikan dan mengamati permasalahan berikut.\\

\textbf{Masalah 1}

Seorang nelayan melihat seekor lumba-lumba sedang berenang mengikuti kecepatan perahu mereka. Lumba-lumba tersebut berenang cepat, terkadang menyelam dan tiba-tiba melayang ke permukakaan air laut. Pada saat nelayan tersebut melihat lumba-lumba menyelam maka ia akan melihatnya melayang ke permukaan 15 detik kemudian dan kembali ke permukaan air laut setelah 3 detik di udara. Demikan pergerakan lumba-lumba tersebut diamati berperiode dalam beberapa interval waktu pengamatan.

Dari ilustrasi diatas, dapatkah kamu sketsa pergerakan lumba-lumba tersebut dalam 2 periode? Ingat pengertian periode pada pelajaran trigonometri di kelas X. Dapatkah kamu tentukan pada interval waktu berapakah lumbalumba tersebut bergerak naik atau turun? Dapatkah kamu temukan konsep fungsi naik/turun?\\

\textbf{Alternatif Penyelesaian:}

Sketsa pergerakan lumba-lumba dalam pengamatan tertentu

\includegraphics[width=12cm,height=5cm]{naikturun5.png}

Sketsa pergerakan naik/turun lumba-lumba dalam pengamatan tertentu

\includegraphics[width=12cm,height=5cm]{naikturun6.png}

Secara geometri pada sketsa di atas, lumba-lumba bergerak turun di interval 0 < t < 7,5 atau 16,5 < t < 25,5 atau 34,5 < t < 36 dan disebut bergerak naik di interval 7,5 < t < 16,5 atau 25,5 < t < 34,5.
%-------------Sesi 2---------------------------------
Coba kamu amati beberapa garis singgung yang menyinggung kurva di saat fungsi naik atau turun di bawah ini. Garis singgung 1 dan 3 menyinggung kurva pada saat fungsi naik dan garis singgung 2 dan 4 menyinggung kurva pada saat fungsi turun.

Garis singgung di interval fungsi naik/turun

\includegraphics[width=14cm,height=7cm]{naikturun7.png}

Selanjutnya, mari kita bahas hubungan persamaan garis singgung dengan fungsi naik atau turun. Pada konsep
persamaan garis lurus, gradien garis adalah tangen sudut yang dibentuk oleh garis itu sendiri dengan sumbu x positif.Pada persamaan garis singgung, gradien adalah tangen sudut garis tersebut dengan sumbu positif sama dengan nilai turunan pertama di titik singgungnya. Pada gambar di atas, misalkan besar masing-masing sudut adalah 0 < $\propto $1 < 900 < $\propto $2 < 900 < $\propto $3 < 900 < $\propto $4 < 900 sehingga nilai
gradien atau tangen sudut setiap garis singgung ditunjukkan pada tabel berikut:

\begin{center}
\includegraphics{naikturun8.png}
\end{center}

Coba kamu amati Gambar diatas dan Tabel sebelumnya Apakah kamu melihat konsep fungsi naik/turun. Coba kamu perhatikan kesimpulan berikut:

Jika garis singgung menyinggung di grafik fungsi naik maka garis singgung akan membentuk 

sudut terhadap sumbu x positif di kuadran I. Hal ini menyebabkan besar gradien adalah positif 

atau m = f '(x) > 0.

Jika garis singgung menyinggung di grafik fungsi turunmaka garis singgung akan membentuk 

sudut terhadap sumbu x positif di kuadran IV. Hal ini menyebabkan besar gradien adalah negatif 

atau m = f '(x) < 0.

Dengan demikian, dapat kita simpulkan bahwa fungsi f(x) yang dapat diturunkan pada interval I, akan mempunyai kondisi sebagai berikut:

\begin{center}
\includegraphics{naikturun9.png}
\end{center}

Misalkan f adalah fungsi bernilai real dan dapat
diturunkan pada setiap x $\in $I maka

1. Jika f '(x) > 0 maka fungsi selalu naik pada interval I.

2. Jika f '(x) < 0 maka fungsi selalu turun pada interval I.

3. Jika f '(x) $\geqslant $0 maka fungsi tidak pernah turun pada interval I.

4. Jika f '(x) $\leqslant $0 maka fungsi tidak pernah naik pada interval I.

Konsep di atas dapat digunakan jika kita sudah memiliki fungsi yang akan dianalisis. Tetapi banyak kasus seharihari harus dimodelkan terlebih dahulu sebelum dianalisis. Perhatikan kembali permasalahan berikut!\\

\textbf{Masalah:}

Tiga orang anak sedang berlomba melempar buah mangga di ketinggian 10 meter. Mereka berbaris menghadap pohon mangga sejauh 5 meter. Anak pertama akan melempar buah mangga tersebut kemudian akan dilanjutkan dengan anak kedua bila tidak mengenai sasaran. Lintasan lemparan setiap anak membentuk kurva parabola. Lemparan anak pertama mencapai ketinggian 9 meter dan batu jatuh 12 meter dari mereka. Lemparan anak kedua melintas di atas sasaran setinggi 5 meter. Anak ketiga berhasil mengenai sasaran. Tentu saja pemenangnya anak ketiga, bukan?
\\

\textbf{Permasalahan!}

Dapatkah kamu mensketsa lintasan lemparan ketiga anak tersebut? Dapatkah kamu membuat model matematika lintasan lemparan? Dapatkah kamu menentukan interval jarak agar masing-masing lemparan naik atau turun berdasarkan konsep turunan?\\


\textbf{Alternatif Penyelesaian}

\textbf{a. Sketsa Lintasan Lemparan}

Permasalahan di atas dapat kita analisis setelah kita modelkan fungsinya. Misalkan posisi awal mereka melempar adalah posisi titik asal O(0,0) pada koordinat kartesius, sehingga sketsa permasalahan di atas adalah sebagai berikut.

\begin{center}
\includegraphics{naikturun10.png}
\end{center}

\textbf{b. Model Lintasan Lemparan}

Kamu masih ingat konsep fungsi kuadrat, bukan? Ingat
kembali konsep fungsi kuadrat yang melalui titik puncak
P($x_{p}, y_{p}) $dan titik sembarang P(x, y) adalah y – $y_{p} = a(x
$– $x_{p})^2 $sementara fungsi kuadrat yang melalui akar-akar x1,
x2 dan titik sembarang P(x, y) adalah y = a(x – $x_{1})(x $– $x_{2}),
$dengan $x_{p}= \dfrac{x_{1}-x_{2}}{2}
$dan a $\neq $0, a bilangan real. Jadi, model
lintasan lemparan setiap anak tersebut adalah:

\textbf{Lintasan lemparan anak pertama}

Lintasan melalui titik O(0,0) dan puncak $p_{1}$(6,9).

\includegraphics{naikturun11.png}

Fungsi lintasan lemparan anak pertama adalah y = –0,25$x^{2} $+ 3x.\\

\textbf{Lintasan lemparan anak kedua}

Lintasan melalui titik O(0,0) dan puncak $P_{2}$(5,15).

\includegraphics{naikturun12.png}

Fungsi lintasan lemparan anak kedua adalah y = –0,6$x^{2} $+ 6x.\\

\textbf{Lintasan lemparan anak ketiga}

Lintasan melalui titik O(0,0) dan puncak P3(5,10).

\includegraphics{naikturun13.png}

Fungsi lintasan lemparan anak ketiga adalah y = –0,4x2 +
4x.

\section{Persamaan Garis Singgung dan Garis Normal}\index{Persamaan Garis Singgung dan Garis Normal}
%----------------------------------------------------------------------------------------
%	CHAPTER 9
%----------------------------------------------------------------------------------------

\chapterimage{chapter_head_2.pdf} % Chapter heading image

\chapter{Integral Tak Tentu Fungsi Aljabar}

\section{Pengertian Integral Tak Tentu Fungsi Aljabar}\index{Pengertian Integral Tak Tentu Fungsi Aljabar}

\section{Sifat-Sifat Integral Tak Tentu Fungsi Aljabar}\index{Sifat-Sifat Integral Tak Tentu Fungsi Aljabar}

\section{Penerapan Integral Tak Tentu Fungsi Aljabar}\index{Penerapan Integral Tak Tentu Fungsi Aljabar}


%----------------------------------------------------------------------------------------

%	BIBLIOGRAPHY
%----------------------------------------------------------------------------------------

\chapter*{Bibliography}
\addcontentsline{toc}{chapter}{\textcolor{ocre}{Bibliography}}
\section*{Books}
\addcontentsline{toc}{section}{Books}
\printbibliography[heading=bibempty,type=book]
\section*{Articles}
\addcontentsline{toc}{section}{Articles}
\printbibliography[heading=bibempty,type=article]

%----------------------------------------------------------------------------------------
%	INDEX
%----------------------------------------------------------------------------------------

\cleardoublepage
\phantomsection
\setlength{\columnsep}{0.75cm}
\addcontentsline{toc}{chapter}{\textcolor{ocre}{Index}}
\printindex

%----------------------------------------------------------------------------------------

\end{document}
