%%%%%%%%%%%%%%%%%%%%%%%%%%%%%%%%%%%%%%%%%
% The Legrand Orange Book
% LaTeX Template
% Version 2.2 (30/3/17)
%
% This template has been downloaded from:
% http://www.LaTeXTemplates.com
%
% Original author:
% Mathias Legrand (legrand.mathias@gmail.com) with modifications by:
% Vel (vel@latextemplates.com)
%
% License:
% CC BY-NC-SA 3.0 (http://creativecommons.org/licenses/by-nc-sa/3.0/)
%
% Compiling this template:
% This template uses biber for its bibliography and makeindex for its index.
% When you first open the template, compile it from the command line with the 
% commands below to make sure your LaTeX distribution is configured correctly:
%
% 1) pdflatex main
% 2) makeindex main.idx -s StyleInd.ist
% 3) biber main
% 4) pdflatex main x 2
%
% After this, when you wish to update the bibliography/index use the appropriate
% command above and make sure to compile with pdflatex several times 
% afterwards to propagate your changes to the document.
%
% This template also uses a number of packages which may need to be
% updated to the newest versions for the template to compile. It is strongly
% recommended you update your LaTeX distribution if you have any
% compilation errors.
%
% Important note:
% Chapter heading images should have a 2:1 width:height ratio,
% e.g. 920px width and 460px height.
%
%%%%%%%%%%%%%%%%%%%%%%%%%%%%%%%%%%%%%%%%%

%----------------------------------------------------------------------------------------
%	PACKAGES AND OTHER DOCUMENT CONFIGURATIONS
%----------------------------------------------------------------------------------------

\documentclass[11pt,fleqn]{book} % Default font size and left-justified equations

%----------------------------------------------------------------------------------------

\input{structure} % Insert the commands.tex file which contains the majority of the structure behind the template

\begin{document}

%----------------------------------------------------------------------------------------
%	TITLE PAGE
%----------------------------------------------------------------------------------------

\begingroup
\thispagestyle{empty}
\begin{tikzpicture}[remember picture,overlay]
\node[inner sep=0pt] (background) at (current page.center) {\includegraphics[width=\paperwidth]{background}};
\draw (current page.center) node [fill=ocre!30!white,fill opacity=0.6,text opacity=1,inner sep=1cm]{\Huge\centering\bfseries\sffamily\parbox[c][][t]{\paperwidth}{\centering The Search for a Title\\[15pt] % Book title
{\Large A Profound Subtitle}\\[20pt] % Subtitle
{\huge Dr. John Smith}}}; % Author name
\end{tikzpicture}
\vfill
\endgroup

%----------------------------------------------------------------------------------------
%	COPYRIGHT PAGE
%----------------------------------------------------------------------------------------

\newpage
~\vfill
\thispagestyle{empty}

\noindent Copyright \copyright\ 2013 John Smith\\ % Copyright notice

\noindent \textsc{Published by Publisher}\\ % Publisher

\noindent \textsc{book-website.com}\\ % URL

\noindent Licensed under the Creative Commons Attribution-NonCommercial 3.0 Unported License (the ``License''). You may not use this file except in compliance with the License. You may obtain a copy of the License at \url{http://creativecommons.org/licenses/by-nc/3.0}. Unless required by applicable law or agreed to in writing, software distributed under the License is distributed on an \textsc{``as is'' basis, without warranties or conditions of any kind}, either express or implied. See the License for the specific language governing permissions and limitations under the License.\\ % License information

\noindent \textit{First printing, March 2013} % Printing/edition date

%----------------------------------------------------------------------------------------
%	TABLE OF CONTENTS
%----------------------------------------------------------------------------------------

%\usechapterimagefalse % If you don't want to include a chapter image, use this to toggle images off - it can be enabled later with \usechapterimagetrue

\chapterimage{chapter_head_1.pdf} % Table of contents heading image

\pagestyle{empty} % No headers

\tableofcontents % Print the table of contents itself

\cleardoublepage % Forces the first chapter to start on an odd page so it's on the right

\pagestyle{fancy} % Print headers again

%----------------------------------------------------------------------------------------
%	PART
%----------------------------------------------------------------------------------------

\part{Part One}

%----------------------------------------------------------------------------------------
%	CHAPTER 1
%----------------------------------------------------------------------------------------

\chapterimage{chapter_head_2.pdf} % Chapter heading image

\chapter{Logika Matematika}

\section{Pernyataan Berkuantor}\index{Pernyataan Berkuantor}

Kuantor dari suatu pernyataan adalah istilah yang digunakan untuk menyatakan “berapa banyak” objek di dalam suatu kalimat atau pembicaraan. Selain untuk menyatakan kuantifikasi, kuantor juga biasa digunakan untuk mengubah kalimat terbuka menjadi suatu kalimat deklaratif.

Definisi : Suatu fungsi pernyataan adalah suatu kalimat terbuka di dalam semesta pembicaraan (semesta pembicaraan diberikan secara eksplisit atau implisit).
Perhatikan dua pernyataan berikut:
1.    Semua planet dalam sistem tata surya mengelilingi matahari.
2.    Ada ikan di laut yang menyusui.
Pernyataan yang mengandung kata semua atau setiap seperti pada pernyataan (1) disebut pernyataan berkuantor universal (kuantor umum). Ungkapan untuk semua atau untuk setiap, disebut kuantor universal atau kuantor umum. Sedangkan pernyataan yang mengandung kata ada atau beberapa seperti pada pernyataan (2) disebut pernyataan berkuantor eksistensial (kuantor khusus). Ungkapan beberapa atau ada disebut kuantor eksistensial atau kuantor khusus.

%------------------------------------------------

\section{Pernyataan Penyangkal (Lingkaran)}\index{Pernyataan Penyangkal  (Lingkaran)}

Dari sebuah pernyataan tunggal (atau majemuk), kita bisa membuat sebuah pernyataan baru berupa “ingkaran” dari pernyataan itu. “ingkaran” disebut juga “negasi” atau “penyangkalan”. Ingkaran menggunakan operasi uner (monar) “” atau “”.

Jika suatu pernyataan p benar, maka negasinya p salah, dan jika sebaliknya pernyataan p salah, maka negasinya p benar.

Perhatikan cara membuat ingkaran dari sebuah pernyataan serta menentukan nilai kebenarannya!

1. p         : kayu memuai bila dipanaskan (S)

~ p      : kayu tidak memuai bila dipanaskan (B)

2.  r          : 3 bilangan positif (B)

~ r        : (cara mengingkar seperti ini salah)

3 bilangan negative

(Seharusnya) 3 bukan bilangan positif  (S)

Nilai kebenaran

Jika p suatu pernyataan benilai benar, maka  ~p bernilai salah dan sebaliknya jika p bernilai salah maka ~p bernilai benar

Konjungsi 

Gabungan  dua  pernyataan  tunggal  yang  menggunakan  kata penghubung  “dan”  sehingga  terbentuk  pernyataan majemuk  disebut konjungsi. Konjungsi mempunyai kemiripan dengan operasi irisan () pada  himpunan.  Sehingga  sifat-sifat  irisan  dapat  digunakan  untuk mempelajari  bagian  ini.



%------------------------------------------------

\section{Penarikan Kesimpulan}\index{Penarikan Kesimpulan}

Lists are useful to present information in a concise and/or ordered way\footnote{Footnote example...}.

\subsection{Numbered List}\index{Lists!Numbered List}

\begin{enumerate}
\item The first item
\item The second item
\item The third item
\end{enumerate}

\subsection{Bullet Points}\index{Lists!Bullet Points}

\begin{itemize}
\item The first item
\item The second item
\item The third item
\end{itemize}

\subsection{Descriptions and Definitions}\index{Lists!Descriptions and Definitions}

\begin{description}
\item[Name] Description
\item[Word] Definition
\item[Comment] Elaboration
\end{description}

%----------------------------------------------------------------------------------------
%	CHAPTER 2
%----------------------------------------------------------------------------------------

\chapter{Induksi Matematika}

\section{Metode Pembuktian Langsung dan Tidak Langsung}\index{Metode Pembuktian Langsung dan Tidak Langsung}

This is an example of theorems.

\subsection{Several equations}\index{Theorems!Several Equations}
This is a theorem consisting of several equations.

\begin{theorem}[Name of the theorem]
In $E=\mathbb{R}^n$ all norms are equivalent. It has the properties:
\begin{align}
& \big| ||\mathbf{x}|| - ||\mathbf{y}|| \big|\leq || \mathbf{x}- \mathbf{y}||\\
&  ||\sum_{i=1}^n\mathbf{x}_i||\leq \sum_{i=1}^n||\mathbf{x}_i||\quad\text{where $n$ is a finite integer}
\end{align}
\end{theorem}

\subsection{Single Line}\index{Theorems!Single Line}
This is a theorem consisting of just one line.

\begin{theorem}
A set $\mathcal{D}(G)$ in dense in $L^2(G)$, $|\cdot|_0$. 
\end{theorem}

%------------------------------------------------

\section{Kontradiksi}\index{Kontradiksi}

This is an example of a definition. A definition could be mathematical or it could define a concept.

\begin{definition}[Definition name]
Given a vector space $E$, a norm on $E$ is an application, denoted $||\cdot||$, $E$ in $\mathbb{R}^+=[0,+\infty[$ such that:
\begin{align}
& ||\mathbf{x}||=0\ \Rightarrow\ \mathbf{x}=\mathbf{0}\\
& ||\lambda \mathbf{x}||=|\lambda|\cdot ||\mathbf{x}||\\
& ||\mathbf{x}+\mathbf{y}||\leq ||\mathbf{x}||+||\mathbf{y}||
\end{align}
\end{definition}

%------------------------------------------------

\section{Induksi Matematis}\index{Induksi Matematis}

\begin{notation}
Given an open subset $G$ of $\mathbb{R}^n$, the set of functions $\varphi$ are:
\begin{enumerate}
\item Bounded support $G$;
\item Infinitely differentiable;
\end{enumerate}
a vector space is denoted by $\mathcal{D}(G)$. 
\end{notation}

%------------------------------------------------

\section{Remarks}\index{Remarks}

This is an example of a remark.

\begin{remark}
The concepts presented here are now in conventional employment in mathematics. Vector spaces are taken over the field $\mathbb{K}=\mathbb{R}$, however, established properties are easily extended to $\mathbb{K}=\mathbb{C}$.
\end{remark}

%------------------------------------------------

\section{Corollaries}\index{Corollaries}

This is an example of a corollary.

\begin{corollary}[Corollary name]
The concepts presented here are now in conventional employment in mathematics. Vector spaces are taken over the field $\mathbb{K}=\mathbb{R}$, however, established properties are easily extended to $\mathbb{K}=\mathbb{C}$.
\end{corollary}

%------------------------------------------------

\section{Kontradiksi}\index{Kontradiksi}

This is an example of propositions.

\subsection{Several equations}\index{Propositions!Several Equations}

\begin{proposition}[Proposition name]
It has the properties:
\begin{align}
& \big| ||\mathbf{x}|| - ||\mathbf{y}|| \big|\leq || \mathbf{x}- \mathbf{y}||\\
&  ||\sum_{i=1}^n\mathbf{x}_i||\leq \sum_{i=1}^n||\mathbf{x}_i||\quad\text{where $n$ is a finite integer}
\end{align}
\end{proposition}

\subsection{Single Line}\index{Propositions!Single Line}

\begin{proposition} 
Let $f,g\in L^2(G)$; if $\forall \varphi\in\mathcal{D}(G)$, $(f,\varphi)_0=(g,\varphi)_0$ then $f = g$. 
\end{proposition}

%------------------------------------------------

\section{Examples}\index{Examples}

This is an example of examples.

\subsection{Equation and Text}\index{Examples!Equation and Text}

\begin{example}
Let $G=\{x\in\mathbb{R}^2:|x|<3\}$ and denoted by: $x^0=(1,1)$; consider the function:
\begin{equation}
f(x)=\left\{\begin{aligned} & \mathrm{e}^{|x|} & & \text{si $|x-x^0|\leq 1/2$}\\
& 0 & & \text{si $|x-x^0|> 1/2$}\end{aligned}\right.
\end{equation}
The function $f$ has bounded support, we can take $A=\{x\in\mathbb{R}^2:|x-x^0|\leq 1/2+\epsilon\}$ for all $\epsilon\in\intoo{0}{5/2-\sqrt{2}}$.
\end{example}

\subsection{Paragraph of Text}\index{Examples!Paragraph of Text}

\begin{example}[Example name]
\lipsum[2]
\end{example}

%------------------------------------------------

\section{Exercises}\index{Exercises}

This is an example of an exercise.

\begin{exercise}
This is a good place to ask a question to test learning progress or further cement ideas into students' minds.
\end{exercise}

%------------------------------------------------

\section{Problems}\index{Problems}

\begin{problem}
What is the average airspeed velocity of an unladen swallow?
\end{problem}

%------------------------------------------------

\section{Vocabulary}\index{Vocabulary}

Define a word to improve a students' vocabulary.

\begin{vocabulary}[Word]
Definition of word.
\end{vocabulary}
%----------------------------------------------------------------------------------------
%	CHAPTER 3
%----------------------------------------------------------------------------------------

\chapterimage{chapter_head_2.pdf} % Chapter heading image

\chapter{Pertidaksamaan Linear Dua Variabel}

\section{Pengertian Pertidaksamaan Linear Dua Variabel}\index{Pengertian Pertidaksamaan Linear Dua Variabel}

\section{Penerapan Pertidaksamaan Linear Dua Variabel}\index{Penerapan Pertidaksamaan Linear Dua Variabel}

%----------------------------------------------------------------------------------------
%	CHAPTER 4
%----------------------------------------------------------------------------------------

\chapterimage{chapter_head_2.pdf} % Chapter heading image

\chapter{Program Linear Dua Variabel}

\section{Pengertian Program Linear Dua Variabel}\index{Pengertian Program Linear Dua Variabel}

\section{Sistem Pertidaksamaan Linear Dua Variabel}\index{Sistem Pertidaksamaan Linear Dua Variabel}

\section{Nilai Optimum Fungsi Objektif}\index{Nilai Optimum Fungsi Objektif}

\section{Penerapan Program Linier Dua Variabel}\index{Penerapan Program Linier Dua Variabel}

%----------------------------------------------------------------------------------------
%	CHAPTER 5
%----------------------------------------------------------------------------------------

\chapterimage{chapter_head_2.pdf} % Chapter heading image

\chapter{Matriks}

\section{Pengertian Matriks}\index{Pengertian Matriks}

\section{Operasi Matriks}\index{Operasi Matriks}

\section{Determinan dan Invers Matriks Berorde 2x2 dan 3x3 }\index{Determinan dan Invers Matriks Berorde 2x2 dan 3x3 }

\section{Pemakaian Matriks Pada Pransformasi Geometri}\index{Pemakaian Matriks Pada Pransformasi Geometri}

%----------------------------------------------------------------------------------------
%	CHAPTER 6
%----------------------------------------------------------------------------------------

\chapterimage{chapter_head_2.pdf} % Chapter heading image

\chapter{Barisan dan Deret}

\section{Pola Bilangan}\index{Pola Bilangan}

\section{Barisan dan Deret Aritmatika}\index{Barisan dan Deret Aritmatika}
Pengertian Barisan Aritmatika
Sebelum memahami pengertian barisan aritmatika kita harus mengetahui terlebih dahulumengenai pengertian basiran bilangan. Barisan bilangan merupakan sebuah urutan dari bilangan yang dibentuk dengan berdasarkan kepada aturan-aturan tertentu. Edangkan barisan aritmetika dapat didefinisikan sebagai suatu barisan bilangan yang tiap-tiap pasangan suku yang berurutan mengandung nilai selisih yang sama persis, contohnya adalah barisan bilangan: 2, 4 , 6, 8, 10, 12, 14, …

Barisan bilangan tersebut dapat disebut sebagai barisana aritmatika karena masing-masing suku memiliki selisih yang sama yaitu 2. Nilai selisih yang muncul pada barisan aritmatika biasa dilambangkan dengan menggunakan huruf b. Setiap bilangan yang membentuk urutan suatu barisan aritmatika disebut dengan suku. Suku ke n dari sebuah barisan aritmatika dapat disimbolkan dengan lambang Un jadi untuk menuliskan suku ke 3 dari sebuah barisan kita dapat menulis U3. Namun, ada pengecualian khusus untuk suku pertama di dalam sebuah barisan bilangan, suku pertama disimbolkan dengan menggunakan huruf a.

Maka, secara umum suatu barian aritmatika memiliki bentuk :

U1,U2,U3,U4,U5,…Un-1
a, atb, a+2b, a+3b, a+4b,…a+(n-1)b

Cara Menentukan Rumus suku ke-n dari Sebuah Barisan
Pada barisan aritmatika, mencaru rumus suku ke-n menjadi lebih mudah karena memiliki nilai selisih yang sama, sehingga rumusnya adalah:

U2 = a + b
U3 = u2 + b = (a + b) + b = a + 2b
U4 = u3 + b = (a + 2b) + b = a + 3b
U5 = u4 + b = (a + 3b) + b = a + 4b
U6 = u5 + b = (a + 4b) + b = a + 5b
U7 = u6 + b = (a + 5b) + b = a + 6b
.
\section{Barisan dan Deret Geometri}\index{Barisan dan Deret Geometri}

%----------------------------------------------------------------------------------------
%	PART
%----------------------------------------------------------------------------------------

\part{Part Two}

%----------------------------------------------------------------------------------------
%	CHAPTER 7
%----------------------------------------------------------------------------------------

\chapterimage{chapter_head_1.pdf} % Chapter heading image

\chapter{Limit Fungsi Aljabar}

\section{Table}\index{Table}

\begin{table}[h]
\centering
\begin{tabular}{l l l}
\toprule
\textbf{Treatments} & \textbf{Response 1} & \textbf{Response 2}\\
\midrule
Treatment 1 & 0.0003262 & 0.562 \\
Treatment 2 & 0.0015681 & 0.910 \\
Treatment 3 & 0.0009271 & 0.296 \\
\bottomrule
\end{tabular}
\caption{Table caption}
\end{table}

%------------------------------------------------

\section{Figure}\index{Figure}

\begin{figure}[h]
\centering\includegraphics[scale=0.5]{placeholder}
\caption{Figure caption}
\end{figure}

%----------------------------------------------------------------------------------------
%	CHAPTER 8
%----------------------------------------------------------------------------------------

\chapterimage{chapter_head_2.pdf} % Chapter heading image

\chapter{Turunan Fungsi Aljabar}

\section{Pengertian Turunan}\index{Pengertian Turunan}

\section{Sifat-Sifat Turunan Fungsi Aljabar}\index{Sifat-Sifat Turunan Fungsi Aljabar}

\section{Penerapan Turunan Fungsi Aljabar}\index{Penerapan Turunan Fungsi Aljabar}

\section{Nilai-Nilai Stasioner}\index{Nilai-Nilai Stasioner}

\section{Fungsi Naik dan Fungsi Turun}\index{Fungsi Naik dan Fungsi Turun}

\section{Persamaan Garis Singgung dan Garis Normal}\index{Persamaan Garis Singgung dan Garis Normal}
%----------------------------------------------------------------------------------------
%	CHAPTER 9
%----------------------------------------------------------------------------------------

\chapterimage{chapter_head_2.pdf} % Chapter heading image

\chapter{Integral Tak Tentu Fungsi Aljabar}

\section{Pengertian Integral Tak Tentu Fungsi Aljabar}\index{Pengertian Integral Tak Tentu Fungsi Aljabar}

\section{Sifat-Sifat Integral Tak Tentu Fungsi Aljabar}\index{Sifat-Sifat Integral Tak Tentu Fungsi Aljabar}

\section{Penerapan Integral Tak Tentu Fungsi Aljabar}\index{Penerapan Integral Tak Tentu Fungsi Aljabar}


%----------------------------------------------------------------------------------------

%	BIBLIOGRAPHY
%----------------------------------------------------------------------------------------

\chapter*{Bibliography}
\addcontentsline{toc}{chapter}{\textcolor{ocre}{Bibliography}}
\section*{Books}
\addcontentsline{toc}{section}{Books}
\printbibliography[heading=bibempty,type=book]
\section*{Articles}
\addcontentsline{toc}{section}{Articles}
\printbibliography[heading=bibempty,type=article]

%----------------------------------------------------------------------------------------
%	INDEX
%----------------------------------------------------------------------------------------

\cleardoublepage
\phantomsection
\setlength{\columnsep}{0.75cm}
\addcontentsline{toc}{chapter}{\textcolor{ocre}{Index}}
\printindex

%----------------------------------------------------------------------------------------

\end{document}
