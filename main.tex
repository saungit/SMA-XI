%%%%%%%%%%%%%%%%%%%%%%%%%%%%%%%%%%%%%%%%%
% The Legrand Orange Book
% LaTeX Template
% Version 2.2 (30/3/17)
%
% This template has been downloaded from:
% http://www.LaTeXTemplates.com
%
% Original author:
% Mathias Legrand (legrand.mathias@gmail.com) with modifications by:
% Vel (vel@latextemplates.com)
%
% License:
% CC BY-NC-SA 3.0 (http://creativecommons.org/licenses/by-nc-sa/3.0/)
%
% Compiling this template:
% This template uses biber for its bibliography and makeindex for its index.
% When you first open the template, compile it from the command line with the 
% commands below to make sure your LaTeX distribution is configured correctly:
%
% 1) pdflatex main
% 2) makeindex main.idx -s StyleInd.ist
% 3) biber main
% 4) pdflatex main x 2
%
% After this, when you wish to update the bibliography/index use the appropriate
% command above and make sure to compile with pdflatex several times 
% afterwards to propagate your changes to the document.
%
% This template also uses a number of packages which may need to be
% updated to the newest versions for the template to compile. It is strongly
% recommended you update your LaTeX distribution if you have any
% compilation errors.
%
% Important note:
% Chapter heading images should have a 2:1 width:height ratio,
% e.g. 920px width and 460px height.
%
%%%%%%%%%%%%%%%%%%%%%%%%%%%%%%%%%%%%%%%%%

%----------------------------------------------------------------------------------------
%	PACKAGES AND OTHER DOCUMENT CONFIGURATIONS
%----------------------------------------------------------------------------------------

\documentclass[11pt,fleqn]{book} % Default font size and left-justified equations

%----------------------------------------------------------------------------------------

\input{structure} % Insert the commands.tex file which contains the majority of the structure behind the template

\begin{document}

%----------------------------------------------------------------------------------------
%	TITLE PAGE
%----------------------------------------------------------------------------------------

\begingroup
\thispagestyle{empty}
\begin{tikzpicture}[remember picture,overlay]
\node[inner sep=0pt] (background) at (current page.center) {\includegraphics[width=\paperwidth]{background}};
\draw (current page.center) node [fill=ocre!30!white,fill opacity=0.6,text opacity=1,inner sep=1cm]{\Huge\centering\bfseries\sffamily\parbox[c][][t]{\paperwidth}{\centering The Search for a Title\\[15pt] % Book title
{\Large A Profound Subtitle}\\[20pt] % Subtitle
{\huge Dr. John Smith}}}; % Author name
\end{tikzpicture}
\vfill
\endgroup

%----------------------------------------------------------------------------------------
%	COPYRIGHT PAGE
%----------------------------------------------------------------------------------------

\newpage
~\vfill
\thispagestyle{empty}

\noindent Copyright \copyright\ 2013 John Smith\\ % Copyright notice

\noindent \textsc{Published by Publisher}\\ % Publisher

\noindent \textsc{book-website.com}\\ % URL

\noindent Licensed under the Creative Commons Attribution-NonCommercial 3.0 Unported License (the ``License''). You may not use this file except in compliance with the License. You may obtain a copy of the License at \url{http://creativecommons.org/licenses/by-nc/3.0}. Unless required by applicable law or agreed to in writing, software distributed under the License is distributed on an \textsc{``as is'' basis, without warranties or conditions of any kind}, either express or implied. See the License for the specific language governing permissions and limitations under the License.\\ % License information

\noindent \textit{First printing, March 2013} % Printing/edition date

%----------------------------------------------------------------------------------------
%	TABLE OF CONTENTS
%----------------------------------------------------------------------------------------

%\usechapterimagefalse % If you don't want to include a chapter image, use this to toggle images off - it can be enabled later with \usechapterimagetrue

\chapterimage{chapter_head_1.pdf} % Table of contents heading image

\pagestyle{empty} % No headers

\tableofcontents % Print the table of contents itself

\cleardoublepage % Forces the first chapter to start on an odd page so it's on the right

\pagestyle{fancy} % Print headers again

%----------------------------------------------------------------------------------------
%	PART
%----------------------------------------------------------------------------------------

\part{Part One}

%----------------------------------------------------------------------------------------
%	CHAPTER 1
%----------------------------------------------------------------------------------------

\chapterimage{chapter_head_2.pdf} % Chapter heading image

\chapter{Logika Matematika}

\section{Pernyataan Berkuantor}\index{Pernyataan Berkuantor}

Kuantor dari suatu pernyataan adalah istilah yang digunakan untuk menyatakan “berapa banyak” objek di dalam suatu kalimat atau pembicaraan. Selain untuk menyatakan kuantifikasi, kuantor juga biasa digunakan untuk mengubah kalimat terbuka menjadi suatu kalimat deklaratif.

Definisi : Suatu fungsi pernyataan adalah suatu kalimat terbuka di dalam semesta pembicaraan (semesta pembicaraan diberikan secara eksplisit atau implisit).
Perhatikan dua pernyataan berikut:
1.    Semua planet dalam sistem tata surya mengelilingi matahari.
2.    Ada ikan di laut yang menyusui.
Pernyataan yang mengandung kata semua atau setiap seperti pada pernyataan (1) disebut pernyataan berkuantor universal (kuantor umum). Ungkapan untuk semua atau untuk setiap, disebut kuantor universal atau kuantor umum. Sedangkan pernyataan yang mengandung kata ada atau beberapa seperti pada pernyataan (2) disebut pernyataan berkuantor eksistensial (kuantor khusus). Ungkapan beberapa atau ada disebut kuantor eksistensial atau kuantor khusus.

%------------------------------------------------

\section{Pernyataan Penyangkal (Lingkaran)}\index{Pernyataan Penyangkal  (Lingkaran)}

Dari sebuah pernyataan tunggal (atau majemuk), kita bisa membuat sebuah pernyataan baru berupa “ingkaran” dari pernyataan itu. “ingkaran” disebut juga “negasi” atau “penyangkalan”. Ingkaran menggunakan operasi uner (monar) “” atau “”.

Jika suatu pernyataan p benar, maka negasinya p salah, dan jika sebaliknya pernyataan p salah, maka negasinya p benar.

Perhatikan cara membuat ingkaran dari sebuah pernyataan serta menentukan nilai kebenarannya!

1. p         : kayu memuai bila dipanaskan (S)

~ p      : kayu tidak memuai bila dipanaskan (B)

2.  r          : 3 bilangan positif (B)

~ r        : (cara mengingkar seperti ini salah)

3 bilangan negative

(Seharusnya) 3 bukan bilangan positif  (S)

Nilai kebenaran

Jika p suatu pernyataan benilai benar, maka  ~p bernilai salah dan sebaliknya jika p bernilai salah maka ~p bernilai benar

Konjungsi 

Gabungan  dua  pernyataan  tunggal  yang  menggunakan  kata penghubung  “dan”  sehingga  terbentuk  pernyataan majemuk  disebut konjungsi. Konjungsi mempunyai kemiripan dengan operasi irisan () pada  himpunan.  Sehingga  sifat-sifat  irisan  dapat  digunakan  untuk mempelajari  bagian  ini.



%------------------------------------------------

\section{Penarikan Kesimpulan}\index{Penarikan Kesimpulan}

Lists are useful to present information in a concise and/or ordered way\footnote{Footnote example...}.

\subsection{Numbered List}\index{Lists!Numbered List}

\begin{enumerate}
\item The first item
\item The second item
\item The third item
\end{enumerate}

\subsection{Bullet Points}\index{Lists!Bullet Points}

\begin{itemize}
\item The first item
\item The second item
\item The third item
\end{itemize}

\subsection{Descriptions and Definitions}\index{Lists!Descriptions and Definitions}

\begin{description}
\item[Name] Description
\item[Word] Definition
\item[Comment] Elaboration
\end{description}

%----------------------------------------------------------------------------------------
%	CHAPTER 2
%----------------------------------------------------------------------------------------

\chapter{Induksi Matematika}

\section{Metode Pembuktian Langsung dan Tidak Langsung}\index{Metode Pembuktian Langsung dan Tidak Langsung}

This is an example of theorems.

\subsection{Several equations}\index{Theorems!Several Equations}
This is a theorem consisting of several equations.

\begin{theorem}[Name of the theorem]
In $E=\mathbb{R}^n$ all norms are equivalent. It has the properties:
\begin{align}
& \big| ||\mathbf{x}|| - ||\mathbf{y}|| \big|\leq || \mathbf{x}- \mathbf{y}||\\
&  ||\sum_{i=1}^n\mathbf{x}_i||\leq \sum_{i=1}^n||\mathbf{x}_i||\quad\text{where $n$ is a finite integer}
\end{align}
\end{theorem}

\subsection{Single Line}\index{Theorems!Single Line}
This is a theorem consisting of just one line.

\begin{theorem}
A set $\mathcal{D}(G)$ in dense in $L^2(G)$, $|\cdot|_0$. 
\end{theorem}

%------------------------------------------------

\section{Kontradiksi}\index{Kontradiksi}

This is an example of a definition. A definition could be mathematical or it could define a concept.

\begin{definition}[Definition name]
Given a vector space $E$, a norm on $E$ is an application, denoted $||\cdot||$, $E$ in $\mathbb{R}^+=[0,+\infty[$ such that:
\begin{align}
& ||\mathbf{x}||=0\ \Rightarrow\ \mathbf{x}=\mathbf{0}\\
& ||\lambda \mathbf{x}||=|\lambda|\cdot ||\mathbf{x}||\\
& ||\mathbf{x}+\mathbf{y}||\leq ||\mathbf{x}||+||\mathbf{y}||
\end{align}
\end{definition}

%------------------------------------------------

\section{Induksi Matematis}\index{Induksi Matematis}

\begin{notation}
Given an open subset $G$ of $\mathbb{R}^n$, the set of functions $\varphi$ are:
\begin{enumerate}
\item Bounded support $G$;
\item Infinitely differentiable;
\end{enumerate}
a vector space is denoted by $\mathcal{D}(G)$. 
\end{notation}

%------------------------------------------------

\section{Remarks}\index{Remarks}

This is an example of a remark.

\begin{remark}
The concepts presented here are now in conventional employment in mathematics. Vector spaces are taken over the field $\mathbb{K}=\mathbb{R}$, however, established properties are easily extended to $\mathbb{K}=\mathbb{C}$.
\end{remark}

%------------------------------------------------

\section{Corollaries}\index{Corollaries}

This is an example of a corollary.

\begin{corollary}[Corollary name]
The concepts presented here are now in conventional employment in mathematics. Vector spaces are taken over the field $\mathbb{K}=\mathbb{R}$, however, established properties are easily extended to $\mathbb{K}=\mathbb{C}$.
\end{corollary}

%------------------------------------------------

\section{Kontradiksi}\index{Kontradiksi}

This is an example of propositions.

\subsection{Several equations}\index{Propositions!Several Equations}

\begin{proposition}[Proposition name]
It has the properties:
\begin{align}
& \big| ||\mathbf{x}|| - ||\mathbf{y}|| \big|\leq || \mathbf{x}- \mathbf{y}||\\
&  ||\sum_{i=1}^n\mathbf{x}_i||\leq \sum_{i=1}^n||\mathbf{x}_i||\quad\text{where $n$ is a finite integer}
\end{align}
\end{proposition}

\subsection{Single Line}\index{Propositions!Single Line}

\begin{proposition} 
Let $f,g\in L^2(G)$; if $\forall \varphi\in\mathcal{D}(G)$, $(f,\varphi)_0=(g,\varphi)_0$ then $f = g$. 
\end{proposition}

%------------------------------------------------

\section{Examples}\index{Examples}

This is an example of examples.

\subsection{Equation and Text}\index{Examples!Equation and Text}

\begin{example}
Let $G=\{x\in\mathbb{R}^2:|x|<3\}$ and denoted by: $x^0=(1,1)$; consider the function:
\begin{equation}
f(x)=\left\{\begin{aligned} & \mathrm{e}^{|x|} & & \text{si $|x-x^0|\leq 1/2$}\\
& 0 & & \text{si $|x-x^0|> 1/2$}\end{aligned}\right.
\end{equation}
The function $f$ has bounded support, we can take $A=\{x\in\mathbb{R}^2:|x-x^0|\leq 1/2+\epsilon\}$ for all $\epsilon\in\intoo{0}{5/2-\sqrt{2}}$.
\end{example}

\subsection{Paragraph of Text}\index{Examples!Paragraph of Text}

\begin{example}[Example name]
\lipsum[2]
\end{example}

%------------------------------------------------

\section{Exercises}\index{Exercises}

This is an example of an exercise.

\begin{exercise}
This is a good place to ask a question to test learning progress or further cement ideas into students' minds.
\end{exercise}

%------------------------------------------------

\section{Problems}\index{Problems}

\begin{problem}
What is the average airspeed velocity of an unladen swallow?
\end{problem}

%------------------------------------------------

\section{Vocabulary}\index{Vocabulary}

Define a word to improve a students' vocabulary.

\begin{vocabulary}[Word]
Definition of word.
\end{vocabulary}
%----------------------------------------------------------------------------------------
%	CHAPTER 3
%----------------------------------------------------------------------------------------

\chapterimage{chapter_head_2.pdf} % Chapter heading image

\chapter{Pertidaksamaan Linear Dua Variabel}

\section{Pengertian Pertidaksamaan Linear Dua Variabel}\index{Pengertian Pertidaksamaan Linear Dua Variabel}

\section{Penerapan Pertidaksamaan Linear Dua Variabel}\index{Penerapan Pertidaksamaan Linear Dua Variabel}

%----------------------------------------------------------------------------------------
%	CHAPTER 4
%----------------------------------------------------------------------------------------

\chapterimage{chapter_head_2.pdf} % Chapter heading image

\chapter{Program Linear Dua Variabel}

\section{Pengertian Program Linear Dua Variabel}\index{Pengertian Program Linear Dua Variabel}

\section{Sistem Pertidaksamaan Linear Dua Variabel}\index{Sistem Pertidaksamaan Linear Dua Variabel}

\section{Nilai Optimum Fungsi Objektif}\index{Nilai Optimum Fungsi Objektif}

\section{Penerapan Program Linier Dua Variabel}\index{Penerapan Program Linier Dua Variabel}

%----------------------------------------------------------------------------------------
%	CHAPTER 5
%----------------------------------------------------------------------------------------

\chapterimage{chapter_head_2.pdf} % Chapter heading image

\chapter{Matriks}

\section{Pengertian Matriks}\index{Pengertian Matriks}

\section{Operasi Matriks}\index{Operasi Matriks}

\section{Determinan dan Invers Matriks Berorde 2x2 dan 3x3 }\index{Determinan dan Invers Matriks Berorde 2x2 dan 3x3 }

\section{Pemakaian Matriks Pada Pransformasi Geometri}\index{Pemakaian Matriks Pada Pransformasi Geometri}

%----------------------------------------------------------------------------------------
%	CHAPTER 6
%----------------------------------------------------------------------------------------

\chapterimage{chapter_head_2.pdf} % Chapter heading image

\chapter{Barisan dan Deret}

\section{Pola Bilangan}\index{Pola Bilangan}

\section{Barisan dan Deret Aritmatika}\index{Barisan dan Deret Aritmatika}
Pengertian Barisan Aritmatika
Sebelum memahami pengertian barisan aritmatika kita harus mengetahui terlebih dahulumengenai pengertian basiran bilangan. Barisan bilangan merupakan sebuah urutan dari bilangan yang dibentuk dengan berdasarkan kepada aturan-aturan tertentu. Edangkan barisan aritmetika dapat didefinisikan sebagai suatu barisan bilangan yang tiap-tiap pasangan suku yang berurutan mengandung nilai selisih yang sama persis, contohnya adalah barisan bilangan: 2, 4 , 6, 8, 10, 12, 14, …

Barisan bilangan tersebut dapat disebut sebagai barisana aritmatika karena masing-masing suku memiliki selisih yang sama yaitu 2. Nilai selisih yang muncul pada barisan aritmatika biasa dilambangkan dengan menggunakan huruf b. Setiap bilangan yang membentuk urutan suatu barisan aritmatika disebut dengan suku. Suku ke n dari sebuah barisan aritmatika dapat disimbolkan dengan lambang Un jadi untuk menuliskan suku ke 3 dari sebuah barisan kita dapat menulis U3. Namun, ada pengecualian khusus untuk suku pertama di dalam sebuah barisan bilangan, suku pertama disimbolkan dengan menggunakan huruf a.

Maka, secara umum suatu barian aritmatika memiliki bentuk :

U1,U2,U3,U4,U5,…Un-1
a, atb, a+2b, a+3b, a+4b,…a+(n-1)b

Cara Menentukan Rumus suku ke-n dari Sebuah Barisan
Pada barisan aritmatika, mencaru rumus suku ke-n menjadi lebih mudah karena memiliki nilai selisih yang sama, sehingga rumusnya adalah:

U2 = a + b
U3 = u2 + b = (a + b) + b = a + 2b
U4 = u3 + b = (a + 2b) + b = a + 3b
U5 = u4 + b = (a + 3b) + b = a + 4b
U6 = u5 + b = (a + 4b) + b = a + 5b
U7 = u6 + b = (a + 5b) + b = a + 6b
.
\section{Barisan dan Deret Geometri}\index{Barisan dan Deret Geometri}

%----------------------------------------------------------------------------------------
%	PART
%----------------------------------------------------------------------------------------

\part{Part Two}

%----------------------------------------------------------------------------------------
%	CHAPTER 7
%----------------------------------------------------------------------------------------

\chapterimage{chapter_head_1.pdf} % Chapter heading image

\chapter{Limit Fungsi Aljabar}

\section{Table}\index{Table}

\begin{table}[h]
\centering
\begin{tabular}{l l l}
\toprule
\textbf{Treatments} & \textbf{Response 1} & \textbf{Response 2}\\
\midrule
Treatment 1 & 0.0003262 & 0.562 \\
Treatment 2 & 0.0015681 & 0.910 \\
Treatment 3 & 0.0009271 & 0.296 \\
\bottomrule
\end{tabular}
\caption{Table caption}
\end{table}

%------------------------------------------------

\section{Figure}\index{Figure}

\begin{figure}[h]
\centering\includegraphics[scale=0.5]{placeholder}
\caption{Figure caption}
\end{figure}

%----------------------------------------------------------------------------------------
%	CHAPTER 8
%----------------------------------------------------------------------------------------

\chapterimage{chapter_head_2.pdf} % Chapter heading image

\chapter{Turunan Fungsi Aljabar}

\section{Pengertian Turunan}\index{Pengertian Turunan}

\section{Sifat-Sifat Turunan Fungsi Aljabar}\index{Sifat-Sifat Turunan Fungsi Aljabar}

\section{Penerapan Turunan Fungsi Aljabar}\index{Penerapan Turunan Fungsi Aljabar}

\section{Nilai-Nilai Stasioner}\index{Nilai-Nilai Stasioner}

\section{Aplikasi Turunan}\index{Aplikasi Turunan}
Konsep turunan adalah subjek yang banyak berperan dalam aplikasi matematika di kehidupan sehari-hari di berbagai bidang. Konsep turunan digunakan untuk
menentukan interval fungsi naik/turun, keoptimalan fungsi dan titik belok suatu kurva.
\subsection{Fungsi Naik dan Fungsi Turun}
Coba bayangkan ketika kamu pergi ke plaza atau mall, di sana kita temukan ekskalator atau lift. Gerakan lift dan ekskalator saat naik dapat diilustrasikan sebagai fungsi naik. Demikian juga gerakan lift dan ekskalator saat turun dapat diilustrasikan sebagai fungsi turun. Amatilah beberapa grafik fungsi naik dan turun di bawah ini dan coba tuliskan cirri-ciri fungsi naik dan fungsi turun sebagai ide untuk mendefinisikan fungsi naik dan turun.

Beberapa grafik fungsi turun dari kiri ke kanan

\includegraphics[width=6cm,height=3cm]{naikturun1.png}

Beberapa grafik fungsi naik dari kiri ke kanan

\includegraphics[width=6cm,height=3cm]{naikturun2.png}

Dari beberapa contoh grafik fungsi naik dan turun di atas,mari kita definisikan fungsi naik dan turun sebagai berikut.

\includegraphics[width=10cm,height=3cm]{naikturun3.png}

\includegraphics[width=10cm,height=7cm]{naikturun4.png}

\subsection{Aplikasi Turunan dalam Permasalahan Fungsi Naik dan Fungsi Turun}

Mari kita bahas aplikasi turunan dalam permasalahan fungsi naik dan fungsi turun dengan memperhatikan dan mengamati permasalahan berikut.\\

\textbf{Masalah 1}

Seorang nelayan melihat seekor lumba-lumba sedang berenang mengikuti kecepatan perahu mereka. Lumba-lumba tersebut berenang cepat, terkadang menyelam dan tiba-tiba melayang ke permukakaan air laut. Pada saat nelayan tersebut melihat lumba-lumba menyelam maka ia akan melihatnya melayang ke permukaan 15 detik kemudian dan kembali ke permukaan air laut setelah 3 detik di udara. Demikan pergerakan lumba-lumba tersebut diamati berperiode dalam beberapa interval waktu pengamatan.

Dari ilustrasi diatas, dapatkah kamu sketsa pergerakan lumba-lumba tersebut dalam 2 periode? Ingat pengertian periode pada pelajaran trigonometri di kelas X. Dapatkah kamu tentukan pada interval waktu berapakah lumbalumba tersebut bergerak naik atau turun? Dapatkah kamu temukan konsep fungsi naik/turun?\\

\textbf{Alternatif Penyelesaian:}

Sketsa pergerakan lumba-lumba dalam pengamatan tertentu

\includegraphics[width=12cm,height=5cm]{naikturun5.png}

Sketsa pergerakan naik/turun lumba-lumba dalam pengamatan tertentu

\includegraphics[width=12cm,height=5cm]{naikturun6.png}

Secara geometri pada sketsa di atas, lumba-lumba bergerak turun di interval 0 < t < 7,5 atau 16,5 < t < 25,5 atau 34,5 < t < 36 dan disebut bergerak naik di interval 7,5 < t < 16,5 atau 25,5 < t < 34,5.
%-------------Sesi 2---------------------------------
Coba kamu amati beberapa garis singgung yang menyinggung kurva di saat fungsi naik atau turun di bawah ini. Garis singgung 1 dan 3 menyinggung kurva pada saat fungsi naik dan garis singgung 2 dan 4 menyinggung kurva pada saat fungsi turun.

Garis singgung di interval fungsi naik/turun

\includegraphics[width=14cm,height=7cm]{naikturun7.png}

Selanjutnya, mari kita bahas hubungan persamaan garis singgung dengan fungsi naik atau turun. Pada konsep
persamaan garis lurus, gradien garis adalah tangen sudut yang dibentuk oleh garis itu sendiri dengan sumbu x positif.Pada persamaan garis singgung, gradien adalah tangen sudut garis tersebut dengan sumbu positif sama dengan nilai turunan pertama di titik singgungnya. Pada gambar di atas, misalkan besar masing-masing sudut adalah 0 < $\propto $1 < 900 < $\propto $2 < 900 < $\propto $3 < 900 < $\propto $4 < 900 sehingga nilai
gradien atau tangen sudut setiap garis singgung ditunjukkan pada tabel berikut:

\begin{center}
\includegraphics{naikturun8.png}
\end{center}

Coba kamu amati Gambar diatas dan Tabel sebelumnya Apakah kamu melihat konsep fungsi naik/turun. Coba kamu perhatikan kesimpulan berikut:

Jika garis singgung menyinggung di grafik fungsi naik maka garis singgung akan membentuk 

sudut terhadap sumbu x positif di kuadran I. Hal ini menyebabkan besar gradien adalah positif 

atau m = f '(x) > 0.

Jika garis singgung menyinggung di grafik fungsi turunmaka garis singgung akan membentuk 

sudut terhadap sumbu x positif di kuadran IV. Hal ini menyebabkan besar gradien adalah negatif 

atau m = f '(x) < 0.

Dengan demikian, dapat kita simpulkan bahwa fungsi f(x) yang dapat diturunkan pada interval I, akan mempunyai kondisi sebagai berikut:

\begin{center}
\includegraphics{naikturun9.png}
\end{center}

Misalkan f adalah fungsi bernilai real dan dapat
diturunkan pada setiap x $\in $I maka

1. Jika f '(x) > 0 maka fungsi selalu naik pada interval I.

2. Jika f '(x) < 0 maka fungsi selalu turun pada interval I.

3. Jika f '(x) $\geqslant $0 maka fungsi tidak pernah turun pada interval I.

4. Jika f '(x) $\leqslant $0 maka fungsi tidak pernah naik pada interval I.

Konsep di atas dapat digunakan jika kita sudah memiliki fungsi yang akan dianalisis. Tetapi banyak kasus seharihari harus dimodelkan terlebih dahulu sebelum dianalisis. Perhatikan kembali permasalahan berikut!\\

\textbf{Masalah:}

Tiga orang anak sedang berlomba melempar buah mangga di ketinggian 10 meter. Mereka berbaris menghadap pohon mangga sejauh 5 meter. Anak pertama akan melempar buah mangga tersebut kemudian akan dilanjutkan dengan anak kedua bila tidak mengenai sasaran. Lintasan lemparan setiap anak membentuk kurva parabola. Lemparan anak pertama mencapai ketinggian 9 meter dan batu jatuh 12 meter dari mereka. Lemparan anak kedua melintas di atas sasaran setinggi 5 meter. Anak ketiga berhasil mengenai sasaran. Tentu saja pemenangnya anak ketiga, bukan?
\\

\textbf{Permasalahan!}

Dapatkah kamu mensketsa lintasan lemparan ketiga anak tersebut? Dapatkah kamu membuat model matematika lintasan lemparan? Dapatkah kamu menentukan interval jarak agar masing-masing lemparan naik atau turun berdasarkan konsep turunan?\\


\textbf{Alternatif Penyelesaian}

\textbf{a. Sketsa Lintasan Lemparan}

Permasalahan di atas dapat kita analisis setelah kita modelkan fungsinya. Misalkan posisi awal mereka melempar adalah posisi titik asal O(0,0) pada koordinat kartesius, sehingga sketsa permasalahan di atas adalah sebagai berikut.

\begin{center}
\includegraphics{naikturun10.png}
\end{center}

\textbf{b. Model Lintasan Lemparan}

Kamu masih ingat konsep fungsi kuadrat, bukan? Ingat
kembali konsep fungsi kuadrat yang melalui titik puncak
P($x_{p}, y_{p}) $dan titik sembarang P(x, y) adalah y – $y_{p} = a(x
$– $x_{p})^2 $sementara fungsi kuadrat yang melalui akar-akar x1,
x2 dan titik sembarang P(x, y) adalah y = a(x – $x_{1})(x $– $x_{2}),
$dengan $x_{p}= \dfrac{x_{1}-x_{2}}{2}
$dan a $\neq $0, a bilangan real. Jadi, model
lintasan lemparan setiap anak tersebut adalah:

\textbf{Lintasan lemparan anak pertama}

Lintasan melalui titik O(0,0) dan puncak $p_{1}$(6,9).

\includegraphics{naikturun11.png}

Fungsi lintasan lemparan anak pertama adalah y = –0,25$x^{2} $+ 3x.\\

\textbf{Lintasan lemparan anak kedua}

Lintasan melalui titik O(0,0) dan puncak $P_{2}$(5,15).

\includegraphics{naikturun12.png}

Fungsi lintasan lemparan anak kedua adalah y = –0,6$x^{2} $+ 6x.\\

\textbf{Lintasan lemparan anak ketiga}

Lintasan melalui titik O(0,0) dan puncak P3(5,10).

\includegraphics{naikturun13.png}

Fungsi lintasan lemparan anak ketiga adalah y = –0,4x2 +
4x.\\

\textbf{c. Interval Fungsi Naik/Turun Fungsi Lintasan}

Coba kamu amati kembali Gambar seketsa lintasan lemparan Secara geometri,jelas kita lihat interval fungsi naik/turun pada masingmasing lintasan, seperti pada tabel berikut:\\

\includegraphics{naikturun14.png}\\

Mari kita tunjukkan kembali interval fungsi naik/turun dengan meng-gunakan konsep turunan yang telah kita pelajari sebelumnya.\\

\textbf{Fungsi naik/turun pada lintasan lemparan anak 1}

Fungsi yang telah diperoleh adalah y = –0,25$x^{2} $+ 3x sehingga y = –0,5$x^{2} $+ 3x. Jadi,

fungsi akan naik: y = –0,5$x^{2} $+ 3x$ \Leftrightarrow $x < 6

fungsi akan turun: y = –0,5x + 3 < 0$ \Leftrightarrow $x > 6

Menurut ilustrasi, batu dilempar dari posisi awal O(0,0) dan jatuh pada posisi akhir Q(12,0) sehingga lintasan lemparan akan naik pada 0 < x < 6 dan turun pada 6 < x < 12.

Bagaimana menunjukkan interval fungsi naik/turun
dengan konsep turunan pada fungsi lintasan lemparan
anak 2 dan anak 3 diserahkan kepadamu.

\textbf {Contoh Soal:} 
Tentukanlah interval fungsi naik/turun fungsi f(x) = $x^{4} $– 2x$^{2}$

\textbf{Alternatif Penyelesaian}

Berdasarkan konsep, sebuah fungsi akan naik jika f '(x) > 0 sehingga:

f '(x) = 4x$^{3} $– 4x > 0$ \Leftrightarrow $4x(x – 1)(x + 1) > 0$ \Leftrightarrow $x = 0 atau x = 1 atau x = –1

Dengan menggunakan interval.

\includegraphics{naikturun15.png}\\

Jadi, kurva fungsi tersebut akan naik pada interval l –1 < x < 0 atau x > 1 tetapi turun pada interval x < –1 atau 0 < x < 1. Perhatikan sketsa kurva f(x) = x$^{4} $– 2x$^{2} $tersebut.\\

\begin{center}
\includegraphics{naikturun16.png}\\
Gambar Fungsi naik/turun kurva f(x) = x4 – 2x2
\end{center}

\subsection{Aplikasi Konsep Turunan dalam
Permasalahan Maksimum dan Minimum}

Setelah menemukan konsep fungsi naik dan turun,
kita akan melanjutkan pembelajaran ke permasalahan
maksimum dan minimum serta titik belok suatu fungsi.
Tentu saja, kita masih melakukan pengamatan terhadap
garis singgung kurva. Aplikasi yang akan dibahas adalah permasalahan titik optimal fungsi dalam interval terbuka dan tertutup, titik belok, dan permasalahan kecepatan maupun percepatan.\\

\textbf{1.Menemukan konsep maksimum dan minimum di interval terbuka}

\textbf{Masalah: }
Seorang anak menarik sebuah tali yang cukup
panjang. Kemudian dia membuat gelombang dari
tali dengan menghentakkan tali tersebut ke atas dan
ke bawah sehingga terbentuk sebuah gelombang
berjalan. Dia terus mengamati gelombang tali yang
dia buat. Dia melihat bahwa gelombang tali memiliki
puncak maksimum maupun minimum. Dapatkah
kamu menemukan konsep nilai maksimum ataupun
minimum dari sebuah fungsi?

\textbf{Penyelesaian :}
Gradien garis singgung adalah tangen sudut yang
dibentuk oleh garis itu sendiri dengan sumbu x positif atau turunan pertama dari titik singgungnya.

\begin{center}
\includegraphics{naikturun17.png}\\
Gambar Sketsa gelombang tali
\end{center}

Coba kamu amati gambar di atas. Garis singgung
(PGS 1, PGS 2, PGS 3 dan PGS 4) adalah garis horizontal atau y = c, c konstan, sehingga gradiennya
adalah m = 0. Keempat garis singgung tersebut menyinggung kurva di titik puncak/optimal, di absis x =$ x_{1}, x = x_{2}, x = x_{3}, dan x = x_{4}$. Dari pengamatan, dapat disimpulkan bahwa sebuah fungsi akan mencapai optimal(maksimum/minimum) pada suatu daerah jika m = f '(x) = 0. Titik yang memenuhi f '(x) = 0 disebut titik stasioner. Berikutnya, kita akan mencoba menemukan hubungan antara titik stasioner dengan turunan kedua fungsi. Pada Gambar sketsa gelombang tali, f '$(x_{1}) = 0, f '(x_{2}) = 0, f '(x_{3}) = 0 dan f '(x_{4}) $= 0. Artinya kurva turunan pertama fungsi melalui sumbu x di titik A($x_{1}, 0), B(x_{2}, 0), C(x_{3}, 0) dan D(x_{4}$, 0).

Coba kamu amati kurva turunan pertama fungsi dan
garis singgungnya sebagai berikut. Kesimpulan apa
yang kamu dapat berikan?

\begin{center}
\includegraphics{naikturun18.png}\\
Gambar Hubungan garis singgung kurva m = f '(x)
dengan titik stasioner
\end{center}

Titik A($x_{1}, y_{1})$ adalah titik maksimum pada Gambar sketsa gelombang tali sehingga titik dengan absis x = x1 adalah titik stasioner karena f '($x_{1})$ = 0. Per-samaan garis singgung kurva dengan gradien M pada fungsi m = f '(x) menyinggung di titik x = $x_{1}$ membentuk sudut di kuadran IV sehingga nilai tangen sudut bernilai negatif. Hal ini mengakibatkan M = m ' = f ''$(x_{1}) $< 0. Dengan kata lain, titik A($x_{1}, y_{1})$ adalah titik maksimum jika f '$(x_{1}) = 0 dan f "(x_{1}$) < 0.

Kesimpulan: Lihat Gambar hubungan garis singgung kurva, misalkan gradien persamaan garis singgung kurva m = f '(x) adalah M sehingga M = m ' = f ''(x) maka hubungan turunan kedua dengan titik stasioner adalah:

\begin{center}
\includegraphics{naikturun19.png}\\
Tabel Hubungan turunan kedua fungsi dengan
titik optimal (stasioner)
\end{center}

\textbf{Sifat:}Misalkan f adalah fungsi bernilai real yang kontinu
dan memiliki turunan pertama dan kedua pada $x_{1} \in I$
sehingga:

1. Jika f '$(x_{1}) = 0 maka titik (x_{1}, f(x_{1}))$disebut stasioner/
kritis

2. Jika f '$(x_{1}) = 0 dan f "(x_{1}) > 0 maka titk (x1, f(x_{1}))$disebut titik balik minimum fungsi

3. Jika f '$(x_{1}) = 0 dan f "(x_{1}) < 0 maka titik (x1, f(x_{1}))$disebut titik balik maksimum fungsi

4. Jika f ''$(x_{1}) = 0 maka titik (x_{1}, f(x_{1})) $disebut titik belok\\

\textbf{Contoh Soal:}Tentukanlah titik balik fungsi kuadrat f(x) = $x^{2} $– 4x + 3

\textbf{Penyelesaian 1 (Berdasarkan Konsep
Fungsi Kuadrat)}

Dengan mengingat kembali pelajaran fungsi kuadrat.
Sebuah fungsi f(x) = $ax^{2} $+ bx + c mempunyai titik balik B(-$\dfrac{b}{2a}$,-$\dfrac{D}{4a}$) dimana fungsi mencapai maksimum untuk a < 0 dan mencapai minimum untuk a > 0 sehingga fungsi
f(x) = $x^{2}$ – 4x + 3 mempunyai titik balik minimum pada B(-$\dfrac{-4}{2(1)}$,-$\dfrac{(-4)^{2}-4(1)(3)}{4(1)}$=B(2,-1).\\

\textbf{Penyelesaian 2 (Berdasarkan Konsep
Turunan)}

Dengan menggunakan konsep turunan di atas maka
fungsi f(x) = $x^{2} $– 4x + 3 mempunyai stasioner: f '(x) = 2x – 4 = 0 atau x = 2 dan dengan mensubstitusi nilai x = 2 ke fungsi y = f(x) = $x^{2} $– 4x + 3 diperoleh y = –1 sehingga titik stasioner adalah B(2, –1). Mari kita periksa jenis keoptimalan fungsi tersebut dengan melihat nilai turunan keduanya pada titik tersebut. f "(x) = 2 atau f "(2) = 2 > 0.
Berdasarkan konsep, titik tersebut adalah titik minimum. Jadi, titik balik fungsi kuadrat f(x) = $x^{2} $– 4x + 3 adalah minimum di B(2, –1).

\begin{center}
\includegraphics{naikturun20.png}\\
Gambar Titik balik fungsi kuadrat f(x) = $x^{2}$ – 4x + 3
\end{center}

\begin{flushleft}
\textbf{Contoh Soal:}

Analisislah kurva fungsi y = f(x) berdasarkan sketsa kurva
turunan pertamanya berikut.
\end{flushleft}

\begin{center}
\includegraphics[width=12cm,height=5cm]{naikturun21.png}\\
Gambar Sketsa turunan pertama suatu fungsi y = f(x)
\end{center}

\begin{flushleft}
\textbf{Alternatif Penyelesaian}

Secara geometri sketsa turunan pertama fungsi di atas,nilai setiap fungsi di bawah sumbu x adalah negatif dan bernilai positif untuk setiap fungsi di atas sumbu x.
\end{flushleft}

\begin{center}
\includegraphics[width=12cm,height=5cm]{naikturun22.png}\\
Gambar Sketsa turunan pertama suatu fungsi y = f(x)
\end{center}

\begin{flushleft}
Dengan demikian, melalui pengamatan dan terhadap grafik turunan pertama dan konsep turunan maka fungsi y = f(x)akan:
\end{flushleft}

• Naik (f '(x) > 0) pada a < x < c, c < x < e dan x > i

• Turun (f '(x) < 0) pada x < a, e < x < g dan g < x < i

• Stasioner (f '(x) = 0) pada absis x = a, x = c, x = e, x = g dan x = i

• Optimal maksimum (f '(x) = 0 dan f "(x) < 0) pada absis x = e

• Optimal minimum (f '(x) = 0 dan f "(x) > 0) pada absis x = a dan x = i.

• Titik belok ( f "(x) = 0) pada absis x = b, x = c, x = d, x = f, x = g dan x = h\\

\textbf{2.Menemukan konsep maksimum dan minimum di interval terbuka}

\begin{flushleft}
\textbf{Contoh Masalah :}

Coba kamu amati posisi titik maksimum dan minimum
dari beberapa gambar berikut.
\end{flushleft}

\begin{center}
\includegraphics{naikturun23.png}\\
Gambar Titik maksimum dan minimum suatu fungsi
\end{center}

\begin{flushleft}
Kesimpulan apa yang kamu peroleh?
\end{flushleft}

\begin{flushleft}
\textbf{Alternatif Penyelesaian :}
\end{flushleft}

Gambar A di atas telah kita bahas pada permasalahan
sebelumnya. Jika kamu amati dengan teliti, perbedaan antara gambar A dengan ketiga gambar lainnya (B, C dan D) adalah terdapat sebuah daerah yang membatasi kurva.Dengan demikian, gambar A adalah posisi titik maksimum/ minimum sebuah fungsi pada daerah terbuka dan ketiga gambar lainnya adalah posisi titik maksimum/minimum sebuah fungsi pada daerah tertutup. Nilai maksimum dan minimum fungsi tidak hanya bergantung pada titik stasioner fungsi tersebut tetapi bergantung juga pada daerah asal fungsi.


\begin{flushleft}
\textbf{Contoh Soal :}
\end{flushleft}

Sebuah pertikel diamati pada interval waktu (dalam menit)tertentu berbentuk kurva f(t) = $t^{3} – 9t^{2} $+ 24t – 16 pada 0 $\leq t \leq 6.$ Tentukanlah nilai optimal pergerakan partikel tersebut.

\begin{flushleft}
\textbf{Alternatif Penyelesaian :}
\end{flushleft}

Daerah asal fungsi adalah {t | 0 $\leq t \leq $6} Titik stasioner
f '(t) = 0

f(t) = $t^{3} – 9t^{2} + $24t – 16 sehingga f '(t) = 3$(t^{2} $– 6t + 8) dan

f "(t) = 6t – 18

f '(t) = 3(t – 2)(t – 4) = 0

t = 2 $\rightarrow f (2) = 4 dan t = 4 \rightarrow$ f(4) = 0

Karena daerah asal {t | 0 $\leq t \leq $6} dan absis t = 2, t = 4 ada dalam daerah asal sehingga:

t = 0 $\rightarrow $f(0) = –16 dan t = 6 $\rightarrow$ f(6) = 20\\

Nilai minimum keempat titik adalah -16 sehingga titik
minimum kurva pada daerah asal adalah A(0,-16) dan
nilai maksimum keempat titik adalah 20 sehingga titik
maksimum kurva pada daerah asal adalah B(6,20).
Perhatikan gambar.

\begin{center}
\includegraphics[width=12cm,height=5cm]{naikturun24.png}\\
Gambar Titik optimal kurva f(t) = $t^{3} – 9t^{2}$ + 24t – 16
untuk 0 $\leq$ t $\leq$ 6.
\end{center}

\begin{flushleft}
\textbf{Contoh Masalah :}
\end{flushleft}

Seorang anak berencana membuat sebuah tabung
dengan alas berbentuk lingkaran tetapi terbuat dari
bahan yang berbeda. Tabung yang akan dibuat harus
mempunyai volume 43.120 cm3. Biaya pembuatan
alas adalah Rp150,- per cm2, biaya pembuatan
selimut tabung adalah Rp80,- per cm2 sementara
biaya pembuatan atap adalah Rp50,- per cm2.
Berapakah biaya minimal yang harus disediakan
anak tersebut?

\begin{flushleft}
\textbf{Alternatif Penyelesaian :}
\end{flushleft}

Mari kita sketsa tabung yang akan dibuat. Misal-kan r
adalah radius alas dan atap tabung, t adalah tinggi tabung $\Pi = \dfrac{22}{7}$.

\begin{center}
\includegraphics{naikturun25.png}\\
Gambar Tabung
\end{center}

V = $\dfrac{22}{7}r^{2}t = 43120 \Leftrightarrow t = \dfrac{7}{22} \times \dfrac{43120}{r^{2}}$

Biaya = (Luas alas $\times$ biaya alas) + (Luas selimut $\times$ biaya selimut) + (Luas atap $\times$ biaya atap)

Biaya =
$\dfrac{22}{7}r^{2} \times 50 + \dfrac{22}{7}rt \times 80 + \dfrac{22}{7}r^{2} \times 50$

Biaya =
$\dfrac{22}{7}r^{2} \times 150 + \dfrac{22}{7}r \times \dfrac{7}{22} \times \dfrac{43120}{r^{2}} \times 80 + \dfrac{22}{7}r^{2} \times 50$

Biaya =
$\dfrac{22}{7}r^{2} \times 200 + \dfrac{43120}{r} \times 80$

\textbf{Biaya B(r) adalah fungsi atas radius r (dalam Rupiah).}

$B(r) = \dfrac{4400}{7}r^{2} + \dfrac{3449600}{r}$

$B'(r) = \dfrac{8800}{7}r - \dfrac{3449600}{r^{2}} = 0$

$\dfrac{88}{7}r^{3} = \dfrac{34496}{r^{2}}$

$r^{3} = 2744 = 143 \Leftrightarrow r = 14$

Jadi biaya minimum

$= \dfrac{22}{7} \times 14^{2} \times 200 + \dfrac{43120}{14} \times 80$

$= 616 \times 200 + 3080 \times 80$

$= 123200 + 246400$

$= 369.600$

Biaya minimum adalah Rp369.600,-
%--------Sesi 3------------------------------------
\section{Persamaan Garis Singgung dan Garis Normal}\index{Persamaan Garis Singgung dan Garis Normal}
%----------------------------------------------------------------------------------------
%	CHAPTER 9
%----------------------------------------------------------------------------------------

\chapterimage{chapter_head_2.pdf} % Chapter heading image

\chapter{Integral Tak Tentu Fungsi Aljabar}

\section{Pengertian Integral Tak Tentu Fungsi Aljabar}\index{Pengertian Integral Tak Tentu Fungsi Aljabar}

\section{Sifat-Sifat Integral Tak Tentu Fungsi Aljabar}\index{Sifat-Sifat Integral Tak Tentu Fungsi Aljabar}

\section{Penerapan Integral Tak Tentu Fungsi Aljabar}\index{Penerapan Integral Tak Tentu Fungsi Aljabar}


%----------------------------------------------------------------------------------------

%	BIBLIOGRAPHY
%----------------------------------------------------------------------------------------

\chapter*{Bibliography}
\addcontentsline{toc}{chapter}{\textcolor{ocre}{Bibliography}}
\section*{Books}
\addcontentsline{toc}{section}{Books}
\printbibliography[heading=bibempty,type=book]
\section*{Articles}
\addcontentsline{toc}{section}{Articles}
\printbibliography[heading=bibempty,type=article]

%----------------------------------------------------------------------------------------
%	INDEX
%----------------------------------------------------------------------------------------

\cleardoublepage
\phantomsection
\setlength{\columnsep}{0.75cm}
\addcontentsline{toc}{chapter}{\textcolor{ocre}{Index}}
\printindex

%----------------------------------------------------------------------------------------

\end{document}
